\section{Introduzione}
Questa relazione descrive lo studio effettuato, i metodi utilizzati ed i risultati raggiunti per la realizzazione dell'elaborato relativo al corso di Analisi delle immagini e dei video, appartenente al corso di laurea specialistica in Ingegneria Informatica di Firenze, tenuto dal Prof. Pietro Pala.

L'elaborato si è incentrato sullo studio di due algoritmi di tracking video, il filtro di Kalman ed il ConDensation, iniziando con l'approfondimento delle rispettive basi teoriche per poi passare all'implementazione di entrambi, finalizzata all'ottenimento di risultati comparativi, che sono stati catalogati ed interpretati.

L'implementazione del software che ha fornito i risultati comparativi è stata effettuata nel linguaggio di programmazione C++ tramite le librerie per il computer vision OpenCV\footnote{Open Source Computer Vision Library http://www.intel.com/technology/computing/opencv/}, sviluppate internamente ad Intel, ma rese pubblicamente fruibili ed utilizzabili tramite una licenza GPL-compatibile; lo sviluppo del codice è stato effettuato sotto controllo di versione Subversion (SVN), in hosting presso Google Code\footnote{http://code.google.com/p/video-tracker/}.


%Rapida descrizione dell'elaborato e di come si articola la relazione, motivazioni della ricerca ecc...