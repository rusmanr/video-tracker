     %%%%%%%%%%%%%%%%%%%%
     %                  %
     %  capitolo1.tex   %
     %                  %
     %%%%%%%%%%%%%%%%%%%%

\chapter{Esperimenti}

\section{Data-set description}
Qui dentro ci va su quali dati abbiamo testato e confrontato i due algoritmi

i video, che tipi di video, video con occlusion, dire che in generale sono macchinine telecomandate, 

\section{Caso 1}
Descrizione del video e dei vari cambi ai parametri effettuati 

per kalman - Q, R

per condensation- n samples

numero frame su cui fare il modulo

verifica presenza nell'ellisse ON/OFF
\section{Caso 2}
Descrizione del video e dei vari cambi ai parametri effettuati
\section{Caso 3}
Descrizione del video e dei vari cambi ai parametri effettuati

mettere ad esempio il nostro caso iniziale (con smooth) per dire che essendo kalman così buono lo si mette in condizioni pessime di lavoro per notare il suo comportamento relativamente alla parte non lineare del moto
\section{Commento dei risultati ottenuti}
qui presumo di vadano gli screenshot pazzi di GNUplot

