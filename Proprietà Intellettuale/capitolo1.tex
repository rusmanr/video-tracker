\chapter{Proprietà Intellettuale e Industriale}
%cosa sono , differenze etc
La sottile differenza lessicale tra le parole ``Proprietà Intellettuale'' e ``Proprietà Industriale'' cela un' enorme differenza che sussite tra le due. Prima di confrontare questi due tipo di \textit{proprietà}, si definisce formalmente quest'ultimo concetto.

\`E bene chiarire che il sostantivo \textit{proprietà}, che accomuna i due concetti, è usato perchè in questo caso vi è facoltà di escludere terzi da qualsiasi uso di un bene che può essere sia materiale che immateriale: tale facoltà è detta diritto di proprietà; in questi casi si tratteranno beni di tipo immateriale.

Per Proprietà Intellettuale si intende la protezione e la valorizzazione delle molteplici forme di creatività intellettuale ed artistica come ad esempio opere musicali, grafiche, editoriali etc. Da un punto di vista giuridico, esse coincidono con tutte le forme di espressione umana e intellettuale che vengono tutelate attraverso il diritto d'autore.

Per Proprietà Industriale invece si definisce l'insieme di istituti giuridici che regolano le creazioni intellettuali umane, ma a contenuto tecnologico; in questo senso si intende opere dell'ingegno (invenzioni) ad ambito industriale. Inoltre si pone la questione sul piano di tutela dei segni distintivi dell'impresa che detiene tale proprietà.

Quindi come si può notare dalle precedenti definizioni appaiono delle analogie sul concetto di \textit{proprietà} e sul diritto al \textit{bene immateriale}, ma si evincono sul piano strutturale ben altre sostanziali differenze: la proprietà intellettuale opera nella più classica tutela dei diritti e quella industriale nella tutela dell'invenzione d'impresa inserita nel contesto di una remunerativa attività economica. Quindi sono entrambe forme di tutela, però rilegate ad ambiti applicativi molto diversi. Per fare chiarezza in ciò il nuovo assetto normativo di riferimento (cioè il Decreto Legislativo n. 30 del 2005) è stato intitolato ``Codice della proprietà industriale''. Esso è un testo unico che raccoglie tutte le norme attinenti al campo dei brevetti e dei marchi. \`E bene sottolineare che resta fuori da questa opera di codificazione la normativa sul diritto d'autore, il cui riferimento è ancora la classica, ma brillante legge 633 del 1941 con le varie modifiche nel corso degli anni. \\
Adesso quindi così come si è differenziato il concetto di ``Proprietà Intellettuale'' e ``Proprietà Industriale'', si  approfondirà i differenti diritti acquisiti con le due proprietà; e si introdurrà rispettivamente, nel seguito di questo capitolo, il diritto d'autore e il titolo di brevetto.

\section{Diritto d'autore e Copyright}
Si va riassumendo in questa sezione il diritto che si ottiene in maniera automatica con la creazione dell'opera, attraverso quella che precedentemente è stata definita come ``Proprietà Intellettuale''. Questa è una sostanziale differenza col brevetto che sarà discussa in \ref{sec:analogDiffBrevetti}.
Nelle scienze giuridiche il diritto d'autore è la posizione giuridica soggettiva dell'autore di un'opera dell'ingegno cui i diversi ordinamenti nazionali e diverse convenzioni internazionali, quale la Convenzione di Berna e la legge vigente, riconoscono la facoltà originaria esclusiva di diffusione e sfruttamento della stessa, ed in ogni caso il diritto ad essere indicato come tale anche quando abbia alienato le facoltà di sfruttamento economico.
Infatti come riporta \textit{l'art.25 della Legge 633/1941} dal diritto d'autore scaturiscono due facoltà che si riassuomo nei seguenti diritti:
\begin{itemize}
 \item Diritti Morali, non cedibili e mai cancellabili. Che si suddividono nei seguenti:

\begin{enumerate}
\item diritto alla paternità dell’opera
\item diritto all’integrit` dell’opera
\item diritto di inedito
\item diritto di ritiro dell’ opera per gravi ragioni morali, cio` il diritto di pentimento.
\end{enumerate}

 \item Diritti Patrimoniali o Economici, possono essere ceduti e terminano dopo il settantesimo anno dalla scomparsa dell'autore.
\end{itemize}


Scendendo nei dettagli, il diritto d'autore è una norma propria degli ordinamenti di \textit{Civil Law} come ad esempio l'Italia, laddove in quelli di \textit{Common Law} si parla di copyright, ad esempio negli gli Stati Uniti d'America.
Per quanto riguarda il copyright vi è da dire che giustamente nel gergo comune si utilizzano spesso indistintamente le espressioni copyright e diritto d’autore. I concetti infatti rimangono di una certà similarità se non per il diverso  sostrato social-economico in cui sono stati ideati.

Si può generalmente dire che il concetto di diritto d’autore è sostanzialmente più vasto del copyright: questo perchè
la matrice del copyright è di stampo anglo-americana, cioè dei cosidetti sistemi Common Law, ed è nato per tutelare l’industria culturale negli States. In questo senso è stato concepito in primis per tutelare l’interesse dell'accordo commerciale soggetto imprenditoriale/autore per la cessione di suddetti diritti.

Il diritto d’autore, partorito invece in Europa, fa un passo in più: l’attenzione della normativa si sposta verso la sfera dell’autore e non verso la sfera del commercio; in questo senso il diritto d’autore risulta più ampio del copyright, perchè aggiunge anche i cosidetti diritti morali incedibli.

Conludendo, poichè la trattazione è incentrata sulla tutela del diritto del software è importante sottolineare che ai diritti patrimoniali sul software sono state dedicate norme specifiche per quanto riguarda la legislazione italiana, ossia gli \textit{artt.64 bis, 64 ter e 64 quarter} della Legge 633/1941 che si conformano alla Direttiva 250/1991/CEE: i diritti esclusivi sui programmi per elaboratori (altresì chiamato \textit{software}) comprendono il diritto di effettuare e autorizzare:

\begin{itemize}
\item la riproduzione temporanea o permanente
\item la traduzione, l’adattamento
\item qualsiasi forma di distribuzione al pubblico compreso la locazione
\end{itemize}

Nonstante il diritto d'autore sia un concetto molto importante in luce anche alla trattazione delle licenza copyleft su cui si basa, si ritiene oppurtuno di dilungare la trattazione maggiormente sugli aspetti relativi alla ``Proprietà Industriale''. Per approfondire e capire maggiormente il concetto di copyright e del rapporto con la gestione dei diritti del software libero, denominata \textit{copyleft}, si consiglia la lettura di \cite[Simone Aliprandi - Teoria e Pratica del copyleft]{Aliprandi-copyleft2}.

Si continuerà quindi la tesi nel riportare i concetti di marchi registrato e brevetto; successivamente si vedrà come queste forme di tutela si applichino ai programmi per elaboratore e come intervenga la gestione alternativa dei diritti secondo license di tipo copyleft.\footnote{In particolare si approfondirà il caso della ricente licenza GPLv3 e delle diatribe collegate.}


\section{Marchi}
Dopo aver sintetizzato estremamente l'idea di base che sta dietro al diritto d'autore/copyright, si viene ad affrontare uno dei due interessanti concetti legati alla sfera della ``Proprietà Industriale''. In questa sezione infatti approfondiremo i marchi \textit{(trademarks)}.

Il trademark è definito come un segno distintivo ed indicativo creato da un individuo o da un impresa che identifica in maniera chiara un tipo di prodotto o servizio nella mente del consumatore: questa particolartà serve per discriminare il prodotto dagli altri. Il marchio è quindi un tipo di proprietà industriale sulla particolarità che mette in evidenza il prodotto/servizio: parole specifiche, nomi propri, disegni, loghi, cifre, suoni, confezioni e anche tonalità cromatiche. Quindi dopo aver registrato il trademark, si ha l'esclusiva commerciale di usare quel tipo di ``logo'' per identificare la propria impresa. Ovviamente questi vantaggi subentrano se le caratteristiche del marchio sono atte a distinguere i prodotti/servizi di un'impresa da quelli delle altre. Nei sistemi Common Law è possibile anche non registrare un marchio, ma conseguentemente sarà possibile solo proteggerlo nella aree geografiche dove viene usato e non oltre. Negli States inoltre quando ci si riferisce ad un marchio registrato che protegge un servizio, viene usato il termine \textit{service mark}.

Anche i marchi, così come i brevetti, trattati nella sezione \ref{sec:brevetti}, devono possedere dei requisiti per essere registrati. Questi requisiti sono riassunti nei seguenti punti:
\begin{description}
 \item[Requisito di novità] un marchio non è nuovo se simile o identico ad un marchio anteriore per prodotto/servizi identici o affini: e cioè per imprese che competono nello stesso mercato.
 \item[Requisito di originalità] il marchio deve essere atto a evidenziare o contraddistinguere il prodotto dell'impresa
 \item[Requisito di lecità] un marchio non deve essere contrario alla legge vigente o all'ordine pubblico e sociale.
 \item[Requisito di verità] un marchio deve trasmettere ai consumatori il vero ambito del prodotto/servizio. Non deve essere ingannevole nei loro confronti. 
 \end{description}

Superati questi requisiti ogni logo può seguire la procedura di refgistrazione secondo le leggi vigenti. C'è però un accordo globale che facilita la tutela internazinale dei trademarks.

Il principale accordo internazione per garantire e facilitare la registrazione di marchi in multiple legislazione risulta l'accordo/protocollo di Madrid. Esso costituisce un sistema centrale di amministrazione per una sicura registrazione dei trademarks estendendo la protezione di una \textit{registrazione internazionale} ottenuta tremie il WIPO \textit{(World Intellectual Property Organization)}. A questo protocollo aderiscono ben 71 stati di tutti il mondo.

La WIPO, in italiano Organizzazione Mondiale per la Proprietà Intellettuale, è una delle agenzie specializzate delle Nazioni Unite ed è stata creata nel 1967 con la finalità di incoraggiare l'attività creativa e promuovere la protezione della proprietà intellettuale nel mondo. Attualmente conta di 183 stati membri, regola 23 trattati internazionali ed ha sede a Ginevra, in Svizzera.

L'ultima accortezza da sottolineare è il sottilce cavillo che vi è tra marchio e designo comunitario. Infatti il disegno comunitario protegge i prodotti dell'industria del design a livello comunitario. Prima tale protezione era ristretta all'ambito nazionale, ma adesso quasi tutti i sistemi nazionali sono stati armonizzati e il design comunitario ha carattere unitario: esercita gli stessi effetti in tutti i paesi dell' Unione Europea presso l' UAMI (Ufficio per l'Armonizzazione nel Mercato Interno).

\subsection{Tipologie di Marchio}
Esistono vari tipi di marchio che possono essere registarti o no. Vengono spiegati nei punti seguenti:
\begin{itemize}
 \item Marchio di fatto. \`E un marchio non registrato che pur non essendo tale gode di una particolare tutela: chi ne ha fatto uso può continuare ad usarlo anche dopo la sua registrazione ottenuta da altri purché il suo uso sia confinato nei limiti territoriali e merceologici antecedenti la registrazione.
 \item Marchio forte/debole.
	\begin{itemize}
	 \item Un marchio forte è quello che ha spiccata originalità e notevole capacità distintiva (ad esempio non deve avere attinenza con il prodotto o servizio a cui si riferisce). Come ad esempio Ferrari (Auto) o Intel (Processori).
	\item Un marchio debole è, invece, quello che presenta una minore originalità (ad esempio per una diretta relazione con il prodotto o servizio che contraddistingue) pur mantenendo una minima capacità distintiva necessaria per differenziarlo ed essere tutelato. Come ad esempio vendita all'ingrosso o dettaglio di articoli sportivi.
	\end{itemize} 

\item Marchio di qualità. Esso ha la funzione di certificare che il prodotto sul quale è apposto abbia determinate caratteristiche qualitative e sia stato prodotto seguendo determinati procedimenti. Qui di seguito sono elencati i principali marchi di qualità:

	\begin{itemize}
	\item Marchio CE. Il Marchio CE attesta che il prodotto su cui è apposto è conforme a tutte le Direttive comunitarie ad esso applicabili.
	\item L'Unione Europea per promuovere e tutelare i prodotti agroalimentari ha creato con il Regolamento CEE n. 2081/92 i seguenti marchi: \textit{DOP} (Denominazione di Origine Protetta), \textit{IGP} (Indicazione Geografica Protetta) e \textit{STG} (Specialità Tradizionale Garantita).
	\end{itemize}

Questa categoria di marchi non deve essere registrata, ma la tutela deriva da apposite leggi introdotte dalla legislazione europea nel 1992 e molto simili ad alcuni sistemi già presenti in alcuni stati europei: in Italia dal 1963 è in vigore \textit{(DOC)} la Denominazione di Origine Controllata per i Vini.\footnote{Curiosità: l'Italia attualmente vanta il primato europeo tra i prodotti DOP, IGP e STG.}



\end{itemize}

\subsection{Marchi Registrati in Italia}
In questa parte si evidenza brevemente quello che costituisce il processo di registrazione di un marchio e i vari tipi di marchi contemplati in Italia.
La tutela di un marchio è disciplinata dagli art. 7 e seguenti del decreto legislativo n. 30 del 10 febbraio 2005: per essere tutelato giuridicamente un marchio deve essere registrato diventando così marchio registrato. La registrazione dura dieci anni a partire dalla data di deposito della domanda, salvo il caso di rinuncia del titolare e alla scadenza può essere rinnovata ogni volta per ulteriori dieci anni.
Essa deve effettuata presso l' UPICA (Ufficio Provinciale Industria Commercio e Artigianato) - sezione Ufficio Brevetti per Invenzioni, Modelli e Marchi - che si trovano presso le Camere di Commercio di ogni Provincia.
Comunque dopo la registrazione il marchio può decadere per le seguenti motivazioni:

\begin{enumerate}
\item  per volgarizzazione, cioè se il marchio sia divenuto nel commercio denominazione generica del prodotto o servizio oppure se abbia perduto la sua capacità distintiva;
\item per illiceità sopravvenuta cioè induca in inganno il pubblico  oppure sia contrario all'ordine pubblico.
\item per non uso, cioè se il titolare del marchio registrato non ne fa un uso effettivo entro cinque anni dalla registrazione o se ne sospende l'uso per un periodo ininterrotto di cinque anni, salvo che il mancato uso non sia giustificato da un motivo legittimo
\end{enumerate}

L'ultimo concetto da trattare a livello giuridico è si sintetizza con la parola  \textit{Licensing}: con questo termine il titolare del marchio concede ad un terzo il diritto di uso del marchio stesso. Di norma i contratti di licensing prevedono il diritto del licenziante di controllare la qualità dei prodotti sui quali il licenziatario appone il marchio.



\section{Brevetti} \label{sec:brevetti}
Il brevetto è lo strumento giuridico che conferisce all'autore di un'invenzione il monopolio temporaneo di sfruttamento dell'invenzione stessa, ossia il diritto di escludere terzi dall'attuare l'invenzione e dal trarne profitto. 

L'invenzione è la forma di protezione più forte che viene concessa a quei trovati che hanno un alto grado di innovazione, ma che, soprattutto, rappresentano una soluzione nuova ed originale ad un problema tecnico.

Il brevetto rappresenta pertanto un monopolio legale, se pur limitato territorialmente e temporalmente. Tale monopolio legale si giustifica con il fatto che il sistema brevettuale è basato su una forma di scambio: il titolare del brevetto riceve protezione per la propria invenzione e in cambio è obbligato a svelare e a descrivere l'invenzione. Le domande di brevetto e i brevetti già concessi sono infatti pubblicati dagli uffici brevetti di tutto il mondo e ciò li rende una primaria fonte di informazione tecnico-scientifica.

Possono costituire oggetto di brevetto i prodotti, i procedimenti produttivi, le varietà vegetali, mentre non sono brevettabili (art. 45 C.P.I.) ``le scoperte, le teorie scientifiche, i metodi matematici, i piani, i principi ed i metodi per attività intellettuale, per gioco o per attività commerciali, i programmi di elaboratori, le presentazioni di informazioni'' in quanto tali. 

Al di là della statica definizione legislativa riuscire a comprendere che cosa possa essere brevettabile come invenzione, richiede molto studio e molta pratica, anche se in modo sintetico si è soliti dire, con una definizione che soddisfa ben poco, che l'invenzione rappresenta una soluzione innovativa ad un problema tecnico; essendo solamente l'idea di fondo del sistema dei brevetti la stessa in tutti i paesi del mondo, si evidenziano profonde differenze nei vari sistemi brevettuali nazionali e continentali, che vanno non solo ad incidere nelle tecniche di brevettazione, ma discriminano anche nell'insieme delle tipologie di invenzioni brevettabili.

Lasciando l'analisi ancora in superficie rispetto alle casistiche particolari delle varie legislazioni, risulta importante chiarire secondo quali requisiti una invenzione è catalogabile come ``brevettabile'':
\begin{description}
 \item[Requisito di novità] L'oggetto del brevetto deve essere nuovo in modo assoluto, cioè non essere mai stato prodotto o brevettato in nessuna parte del mondo. Il concetto di novità viene inteso in senso ampio e si ricomprende nello "stato della tecnica" tutto ciò che è stato reso pubblico, in Italia o all’estero, prima della data di deposito della domanda di brevetto. Risulta chiarificatore un esempio banale per distinguere la brevettabilità dalla possibilità di produrre e/o sfruttare un invenzione: se un oggetto è stato realizzato o brevettato, ad esempio, in Cina ma non in Italia, ciò significa che chiunque in Italia potrà produrlo e venderlo, ma non certo che possa anche brevettarlo: la differenza è evidente, in quanto senza brevetto potrà agire in regime di libera concorrenza, senza pretendere di avere alcun monopolio.
\item [Requisito di originalità] Chiamato anche ``attività inventiva'' o ``non ovvietà'' sussiste ogni volta che l'invenzione non risulta in modo evidente dallo stato della tecnica per una persona esperta del ramo. Stabilire quando un trovato soddisfi questo requisito è estremamente difficoltoso in quanto è richiesto che l'invenzione per essere brevettabile non debba essere banale, ma rappresentare un progresso, un passo in avanti ``non ovvio'' rispetto allo stato della tecnica attuale. Proprio per stabilire quanto appena detto è spesso interessata la giurisprudenza, anche questa molto altalenante sui giudizi sull’argomento, ed è spesso intorno a questo punto che si giocano le cause relative alla nullità di un brevetto. 
\item [Requisito di industrialità]Risultano brevettabili solo soluzioni che possono essere riprodotte a livello industriale, escludendo tutte le applicazioni artigianali o comunque legate ad un contributo rilevante della persona che le ha realizzate.
\item [Requisito di liceità]Non sono brevettabili invece oggetti che possono ledere il senso del buon costume o essere contrarie all'ordine pubblico, concetti questi in continua evoluzione.
 \end{description}

Non da trascurare è l'aspetto legato ai diritti che poi scaturiscono dall'invenzione stessa; in seguito all'invenzione scaturiscono nei confronti dell'autore due tipologie diverse di diritto: il diritto morale sull'invenzione ed il diritto di brevetto. Mentre il primo concerne un'area strettamente personale e non è cedibile, il secondo riguarda lo sfruttamento economico dell'invenzione, e quindi risulta cedibile.
% 	\subsection{Royalties}
\subsection{Il brevetto in Italia}
Nello stato italiano la normativa sui brevetti è stabilita dal Titolo IX del Libro Quinto del Codice Civile, intitolato ``Dei diritti sulle opere dell'ingegno e sulle invenzioni industriali''. 

Nel dettaglio l'articolo specifico è il 2585, che definisce l'oggetto del brevetto come segue:

\textit{``Possono costituire oggetto di brevetto le nuove invenzioni atte ad avere un'applicazione industriale, quali un metodo o un processo di lavorazione industriale, una macchina, uno strumento, un utensile o un dispositivo meccanico, un prodotto o un risultato industriale e l'applicazione tecnica di un principio scientifico, purché essa dia immediati risultati industriali.''}

Rimane da precisare che in Italia tutta la materia gravitante attorno al campo della Proprietà Intelletuale e Brevetti è regolamentata da una apposita legislazione, confluita dal 2005 (con la legislazione sui marchi, modelli, design registrati) nel D.Lgs. 10 Febbraio 2005, chiamato codice della Proprietà Intellettuale.

Riguardo alla brevettabilità non è possibile, per la legislazione italiana, registrare un indice di ciò che è catalogabile come ``brevettabile''; si ritiene comunque che la serie di requisiti richiesti sia sostanzialmente analoga a quella anticipata nella sezione precedente, come possiamo evincere dagli articoli 46 e 48 del Codice della Proprietà Intellettuale. 

Tuttavia sono note e tassative la categorie di eccezioni, ovvero le aree di lavoro e di scoperta non brevettabili. Queste comprendono:
\begin{itemize}
\item le scoperte, le teorie scientifiche e i metodi matematici
\item i piani, i principi e i metodi per attività intellettuali, per gioco o per attività commerciali e i programmi di elaboratori;
\item le presentazioni di informazioni
\item i metodi per il trattamento chirurgico o terapeutico del corpo umano o animale e i metodi di diagnosi applicati al corpo umano o animale; possono però esserlo i prodotti, in particolare sostanze o miscele di sostanze, impiegati per l'attuazione dei metodi diagnostici, terapeutici o chirurgici: non costituisce invenzione il metodo, possono costituirla gli strumenti necessari alla sua applicazione.
\item le razze animali, eccezione fatta per i procedimenti microbiologici\end{itemize}

L'autore dell'invenzione ha il diritto di disporre della stessa e di commericalizzarla. Questi diritti sono intesi come diritti patrimoniali, e sono riconosciuti come propri dell'inventore; l'aggettivo ``patrimoniali'' sottolinea la classe di diritti che possono essere oggetto di cessione, solitamente mediante contratto e a titolo oneroso.

Come già avevamo accennato nella trattazione generale esiste anche nella legislazione italiana il riconoscimento per il diritto morale, che risulta incedibile, intrasmissibile e strettamente personale. 

L'autore dell'invenzione deve richiedere la registrazione del brevetto presso l'ufficio italiano brevetti, che è responsabile della verifica dei requisiti sopracitati e della non contrarietà alle leggi; se tutto è ritenuto regolare si procede alla registrazione del brevetto.

Normalmente la durata della tutela brevettuale è ventennale, ma esistono delle clausole che possono indurre una prescrizione anteriore; se ad esempio il titolare del brevetto non lo rinnova, oppure se entro tre anni dalla registrazione del brevetto l'invenzione non è ancora conclusa. Esistono anche uteriori casistiche, indotte dalla dipendenza di ulteriori brevetti o dalla situazione di inventori dipendenti subordinati di aziende, ma non essendo strettamente inerenti alla trattazione saranno trascurate.

L'azione spettante a chi viola un brevetto industriale è quella di contraffazione.

\subsection{Il brevetto nell'Unione Europea}
Il Brevetto Europeo è una tutela Istituita con la Convenzione di Monaco sul brevetto europeo del 1973, riprendendo le indicazioni della Convenzione di Strasburgo del 1963. 

L'ente responsabile della brevettazione a livello comunitario è l'``Organizzazione europea dei Brevetti'' (in lingua inglese, \textit{European Patent Organisation}, da cui deriva l'acronimo EPO) è un'organizzazione pubblica internazionale creata dalla Convenzione europea dei Brevetti. L'Organizzazione europea dei Brevetti ha sede a Monaco di Baviera, in Germania.

---------- da qui è ancora da rimettere a posto ----------

Sebbene si parli di brevetto europeo come se fosse un titolo unitario, in effetti non è così: sia la domanda che l'esame sono infatti univoci, ma il titolo, una volta rilasciato, diventa una collezione di brevetti nazionali e conferisce al titolare gli stessi diritti che gli verrebbero conferiti dai vari brevetti nazionali degli stati designati.

I brevetti europei sono concessi dopo un'accurata ricerca dello stato della tecnica ed un esame di merito che ne verifica i requisiti di brevettabilità.
\subsection{Il brevetto Internazionale}
\section{Analogie e Differenze}\label{sec:analogDiffBrevetti}
Questa sezione piu che lunga deve essere efficace e sintetica. magari con una tabellina
