
         %%%%%%%%%%%%%%%%%%%%%%%%%%%%%%%%%%%%%%%%
         %                                      %
         %  "Scheletro" di una tesi di laurea   %
         %                   		        %
         %      dell'Universita` di Firenze     %
         %                                      %
         %%%%%%%%%%%%%%%%%%%%%%%%%%%%%%%%%%%%%%%%

              %%%%%%%%%%%%%%%%%%%%%%%%%%%%%%%%%%%%%%%%%%%%%%%%%%%%%%%%%%%%%%%%%%%%%%
              %  Autori: Iacopo Masi e Marco Meoni (riadattata da Gianluca Gorni)  %
              %%%%%%%%%%%%%%%%%%%%%%%%%%%%%%%%%%%%%%%%%%%%%%%%%%%%%%%%%%%%%%%%%%%%%%

   %%%%%%%%%%%%%%%%%%%%%%%%%%%%%%%%%%%%%%%%%%%%%%%%%%
   %%%%%  Ultima modifica:  13 gennaio 2006    %%%%%
   %%%%%%%%%%%%%%%%%%%%%%%%%%%%%%%%%%%%%%%%%%%%%%%%%%


  %%%%%%%%%%%%%%%%%%%%%%%%%%%%%%%%%%%%%%%%%%%%%%%%%%%%%%%%%
  % Usare una versione di LaTeX con sillabazione italiana %
  %%%%%%%%%%%%%%%%%%%%%%%%%%%%%%%%%%%%%%%%%%%%%%%%%%%%%%%%%

\documentclass[12pt,a4paper,oneside,italian]{book}

% Usare "oneside" invece di "twoside"
% nelle bozze, per risparmiare carta:
% "twoside" produce diverse pagine bianche
% alla fine dei capitoli.

                    %%%%%%%%%%%%%%%%%%%%%%%%%%%%%%%%
                    %         inputenc             %
                    %  Usare l'opzione "latin1"    %
		    %  oppure "utf8"   		   %
                    %  se si vogliono scrivere     %
%                   %  lettere accentate da        %
                    %  tastiera su Windows o Unix  %
                    %%%%%%%%%%%%%%%%%%%%%%%%%%%%%%%%
% \usepackage[utf8x]{inputenc}

\usepackage[utf8]{inputenc}
% \usepackage[latin1]{inputenc}
\usepackage{rotating}
\usepackage{fancyvrb}
\usepackage{verbatim}
       %%%%%%%%%%%%%%%%%%%%%%%%%%%%%%%%%%%%%%%%%%%%%%
       %                  babel                     %
       % Pacchetto tipico per una tesi in italiano. %
       %%%%%%%%%%%%%%%%%%%%%%%%%%%%%%%%%%%%%%%%%%%%%%


\usepackage[italian]{babel}

   %%%%%%%%%%%%%%%%%%%%%%%%%%%%%%%%%%%%%%%%%%%%%%%%%%%%%%%%%%%%
   % Se nella tesi si inseriscono dei passi in un'altra       %
   % lingua (inglese, per fissare le idee), si puo' istruire  %
   % il TeX di sillabare quella parte di testo con le regole  %
   % inglesi, invece che italiane. A questo scopo basta       %
   % scrivere                                                 %
   %                                                          %
   %    \usepackage[english,italian]{babel}                   %
   %                                                          %
   % al posto di \usepackage[italian]{babel},                 %
   % dopodiche la sillabazione sara' italiana fintanto che    %
   % non si incontra il comando \selectlanguage{english}.     %
   % Per tornare all'italiano si scrive                       %
   % \selectlanguage{italian}                                 %
   %%%%%%%%%%%%%%%%%%%%%%%%%%%%%%%%%%%%%%%%%%%%%%%%%%%%%%%%%%%%

\usepackage{unifitesi}

% \usepackage{graphicx} % gia' caricato da unifitesi
\graphicspath{{./figure/}}

% Per l'ipertesto:
% \usepackage{hyperref} % gia' caricato da unifitesi
\hypersetup{
  % pdfpagelayout=SinglePage, % default
  % pdfpagemode=UseOutlines, % default
  % bookmarksopen, % default
  % bookmarksopenlevel=2, % default;
  pdftitle=Gestione
della conoscenza e della proprietà intellettuale,
  pdfauthor=Marco Meoni e Iacopo Masi,
  pdfsubject=Elaborato Gestione
della conoscenza e della proprietà intellettuale,
  pdfkeywords=Relazione tesina Ingegneria Informatica Firenze Iacopo Masi Marco Meoni Gestione
della conoscenza e della proprietà intellettuale,} % Queste informazioni non vengono stampate, ma sono conservate nel documento pdf. Sono consultabili col menu "File>Document Properites>Description". Vengono buone a scopi archivistici.
%%%%%%%%%%%%%%%%%%%%%%%%%%%%%%%%%%%%%%%%%%%%%%%%%%%%%%%%%%%%

       %%%%%%%%%%%%%%%%%%%%%%%%%%%%%%%%%%%%%%%%%%%%%%%%
       % Pacchetti tipici per una tesi di matematica  %
       %%%%%%%%%%%%%%%%%%%%%%%%%%%%%%%%%%%%%%%%%%%%%%%%

\usepackage{amsmath,amsfonts,amssymb,amsthm}
\usepackage{latexsym}
%\usepackage{vertbars}
\usepackage{changebar}

%%%%%%%%%%%%%%%%%%%%%%%%%%%%%%%%%%%%%%%%%%%%%%%%%%%%%%%
%                    graphicx                         %
%                                                     %
%   Uno dei pacchetti per l'inserzione di figure      %
%   in formato eps e` "graphicx". Ce ne sono diversi  %
%   altri da cui scegliere.                           %
%                                                     %
%   Esempio di uso: avendo un file di nome            %
%   figura1.eps questa si inserisce nella tesi        %
%   col comando                                       %
%                                                     %
%        \begin{figure}[ht]                           %
%        \begin{center}                               %
%        \includegraphics{figura1.eps}                %
%        \caption[nome breve]{nome lungo}             %
%        \end{center}                                 %
%        \end{figure}                                 %
%                                                     %
%   Il "nome breve" e` quello che apparira`           %
%   nell'indice delle figure ed e' opzionale.         %
%   Il "nome lungo" e' quello che appare              %
%   sotto la figura.                                  %
%   (Ci sono opzioni per scalare, spostare, ruotare   %
%   le figure).                                       %
%   Con \graphicspath{{./figure/}} si dice            %
%   al LaTeX di cercare le figure nella cartella      %
%   "figure" situata allo stesso livello di           %
%   questo documento                                  %
%                                                     %
%%%%%%%%%%%%%%%%%%%%%%%%%%%%%%%%%%%%%%%%%%%%%%%%%%%%%%%

\linespread{1.3}
 %%%%%%%%%%%%%%%%%%%%%%%%%%%%%%%%%%%%%%%%%%%%%%%%%%
 % usate linesprea 1.6 per avere interlinea doppio%
 %%%%%%%%%%%%%%%%%%%%%%%%%%%%%%%%%%%%%%%%%%%%%%%%%%

  %%%%%%%%%%%%%%%%%%%%%%%%%%%%%%%%%%%%%%%%%%%
   %  Esempi di macro definite dall'utente.  %
   %  Le prime definiscono dei comandi per   %
   %  scrivere i caratteri speciali per      %
   %  gli insiemi numerici fondamentali      %
   %  (naturali, interi, razionali, reali,   %
   %  complessi                              %
   %%%%%%%%%%%%%%%%%%%%%%%%%%%%%%%%%%%%%%%%%%%

\newcommand{\N}{\mathbb{N}}
\newcommand{\Z}{\mathbb{Z}}
\newcommand{\Q}{\mathbb{Q}}
\newcommand{\R}{\mathbb{R}}
\newcommand{\C}{\mathbb{C}}


   %%%%%%%%%%%%%%%%%%%%%%%%%%%%%%%%%%%%%%%%%%%%
   %  Delle macro che definiscono operatori   %
   %  non predefiniti in LaTeX. Ogni utente   %
   %  aggiunge quelle che servono. Questi     %
   %  sono solo esempi arbitrari.             %
   %%%%%%%%%%%%%%%%%%%%%%%%%%%%%%%%%%%%%%%%%%%%

\DeclareMathOperator{\traccia}{tr}
\DeclareMathOperator{\sen}{sen}
\DeclareMathOperator{\arcsen}{arcsen}
\DeclareMathOperator*{\maxlim}{max\,lim}
\DeclareMathOperator*{\minlim}{min\,lim}
\DeclareMathOperator*{\deepinf}{\phantom{\makebox[0pt]{p}}inf}

    %%%%%%%%%%%%%%%%%%%%%%%%%%%%%%%%%%%%%%%%%%%%
    % Esempi di macro piu` elaborate,          %
    % contenenti degli argomenti.              %
    % Compongono gli indici delle sommatorie   %
    % e delle produttorie in modo diverso      %
    % da quello standard del TeX. Dovrebbero   %
    % funzionare bene quando gli estremi della %
    % sommatoria sono piccoli. Chi volesse     %
    % usarle estesamente farebbe bene a        %
    % lavorarci sopra.                         %
    %%%%%%%%%%%%%%%%%%%%%%%%%%%%%%%%%%%%%%%%%%%%

\newcommand{\varsum}[3]{\sum_{#2}^{#3}\!
   {\vphantom{\sum}}_{#1}\;}
\newcommand{\varprod}[3]{\sum_{#2}^{#3}\!
   {\vphantom{\sum}}_{#1}\;}

  %%%%%%%%%%%%%%%%%%%%%%%%%%%%%%%%%%%%%%%%%%%%%%%%%%%%%%%
  %          Numerazione delle formule                  %
  % Se non specificato altrimenti, il LaTeX numera le   %
  % formule come (capitolo.formula) (per esempio (2.5)  %
  % e` la quinta formula del secondo capitolo).         %
  % Con le istruzioni seguenti invece la numerazione    %
  % diventa (capitolo.sezione.formula) (per esempio     %
  % (3.2.6) e` la sesta formula della seconda sezione   %
  % del terzo capitolo):                                %
  %%%%%%%%%%%%%%%%%%%%%%%%%%%%%%%%%%%%%%%%%%%%%%%%%%%%%%%

\makeatletter
\@addtoreset{equation}{section}
\makeatother
\renewcommand{\theequation}%
  {\thesection.\arabic{equation}}


              %%%%%%%%%%%%%%%%%%%%%%%%%%
              % Stile degli enunciati  %
              %%%%%%%%%%%%%%%%%%%%%%%%%%

%%%%%%%%%%%%%%%%%%%%%%%%%%%%%%%%%%%%%%%%%%%%%%%%%%%%%%%%%%%
% Con le dichiarazioni seguenti                           %
% teoremi, definizioni, proposizioni, lemmi e corollari   %
% vengono numerati capitolo per capitolo e con un         %
% contatore unico per tutti (per esempio, se subito dopo  %
% il Teorema 2.1 c'e' una definizione, questa sara'       %
% Definizione 2.2)                                        %
%%%%%%%%%%%%%%%%%%%%%%%%%%%%%%%%%%%%%%%%%%%%%%%%%%%%%%%%%%%

\theoremstyle{plain}
\newtheorem{teorema}{Teorema}[chapter]
\newtheorem{proposizione}[teorema]{Proposizione}
\newtheorem{lemma}[teorema]{Lemma}
\newtheorem{corollario}[teorema]{Corollario}

\theoremstyle{definition}
\newtheorem{definizione}[teorema]{Definizione}
\newtheorem{esempio}[teorema]{Esempio}

\theoremstyle{remark}
\newtheorem{osservazione}[teorema]{Osservazione}

  %%%%%%%%%%%%%%%%%%%%%%%%%%%%%%%%%%%%%%%%%%%%%%%%%%%%%%%%
  % I comandi si usano cosi`:                            %
  %                                                      %
  %   \begin{teorema}[di Pitagora]                       %
  %   La somma dei quadrati ecc.                         %
  %   \end{teorema}                                      %
  %                                                      %
  % Le parole "di Pitagora" fra parentesi quadre         %
  % sono facoltative. Non bisogna inserire               %
  % manualmente degli spazi prima e dopo gli enunciati,  %
  % perche' e` automatico!                               %
  %%%%%%%%%%%%%%%%%%%%%%%%%%%%%%%%%%%%%%%%%%%%%%%%%%%%%%%%


  %%%%%%%%%%%%%%%%%%%%%%%%%%%%%%%%%%%%%%%%%%%%%%%%%%%%%%%%%%%%%%
  % Il pacchetto amsthm definisce anche l'ambiente "proof"     %
  % per le dimostrazioni.                                      %
  % Esempio di uso:                                            %
  %                                                            %
  %   \begin{proof}                                            %
  %   Sia $X$ un insieme ecc.                                  %
  %   \end{proof}                                              %
  %                                                            %
  %%%%%%%%%%%%%%%%%%%%%%%%%%%%%%%%%%%%%%%%%%%%%%%%%%%%%%%%%%%%%%

       %%%%%%%%%%%%%%%%%%%%%%%%%%%%%%%%%%%%%%%%%%%%%%%%%%%%%%%
       %                   makeidx                           %
       %                                                     %
       % Pacchetto per la generazione automatica dell'indice %
       % analitico. Per esempio, se vogliamo che la parola   %
       % "analitico" venga indicizzata nella frase           %
       %                                                     %
       %    "un metodo analitico di soluzione"               %
       %                                                     %
       % bisogna scrivere                                    %
       %                                                     %
       %    "un metodo analitico\index{analitico} di         %
       %              soluzione".                            %
       %                                                     %
       % Compilando il file, il LaTeX produrra' un file      %
       % ausiliario che termina con ".idx". Bisogna far      %
       % processare questo file idx dal programma            %
       % ausiliario "bibtex", che produrra' a sua volta un   %
       % altro file ancora. Dare infine un'ultima passata    %
       % col LaTeX. Si puo' tranquillamente lasciare         %
       % la compilazione dell'indice verso la fine della     %
       % stesura del lavoro, quando tutto e' ormai quasi     %
       % definitivo.                                         %
       %                                                     %
       %%%%%%%%%%%%%%%%%%%%%%%%%%%%%%%%%%%%%%%%%%%%%%%%%%%%%%%

\usepackage{makeidx}
\usepackage{array}
\usepackage{tocbibind}
\usepackage{listings}
\usepackage{color}
\makeindex

% Ridefiniamo la riga di testa delle pagine:
\usepackage{fancyhdr}
\pagestyle{fancy}
\renewcommand{\chaptermark}[1]{\markboth{#1}{}}
\renewcommand{\sectionmark}[1]{\markright{\thesection\ #1}}
\fancyhf{}
\fancyhead[LE,RO]{\thepage}
\fancyhead[LO]{\rightmark}
\fancyhead[RE]{\bfseries\leftmark}
\renewcommand{\headrulewidth}{0.1pt}
\renewcommand{\footrulewidth}{0pt}
\headsep=50pt


               %%%%%%%%%%%%%%%%%%%%%%%%%%%%%%%%%%%%%%
               %  Informazioni generali sulla Tesi  %
               %    da usare nell'intestazione      %
               %%%%%%%%%%%%%%%%%%%%%%%%%%%%%%%%%%%%%%
\titolo{Brevetto Software: \\ tutela dell'invenzione o limite all'innovazione?}


 \laureando{Iacopo Masi, Marco Meoni}
  \annoaccademico{2007-2008}
 \facolta{Ingegneria} % (default)
  \corsodilaurea{Ingegneria Informatica} % per la laurea vecchio ordinamento
% \corsodilaureatriennale{Informatica}
% \corsodilaureaspecialistica{Ingegneria}
  \relatore[Prof.]{Gaetano Cascini}
  \correlatore[Avv.]{Marina Da Bormida}
  %\correlatoredue[Ing.]{Leonardo Maccari}

  %\dedica{\textit{``Baby don't cry\\make it funky''} \\Zucchero} % (facoltativo)
%
   %%%                                    %        %%    %%
  %   %                                   %         %     %
  %      %%%  % %%  %%%%   %%%         %%%%  %%%    %     %    %%%%
  %     %   % %%  % %   % %   %       %   % %   %   %     %   %   %
  %     %   % %     %   % %   %       %   % %%%%%   %     %   %   %
  %   % %   % %     %   % %   %       %   % %       %     %   %  %%
   %%%   %%%  %     %%%%   %%%         %%%%  %%%%  %%%   %%%   %% %
                    %
                    %


                          %%%%%               %
                            %
                            %    %%%   %%%%  %%
                            %   %   % %       %
                            %   %%%%%  %%%    %
                            %   %         %   %
                            %    %%%% %%%%   %%%


 \begin{document}

         %%%%%%%%%%%%%%%%%%%%%%%%%%%%%%%%%%%%%%%%%%%%%%%%%
         %            Intestazione                       %
         %                                               %
         % Per l'intestazione completa bisogna           %
         % essersi procurati il file "firenze.eps". %
         %%%%%%%%%%%%%%%%%%%%%%%%%%%%%%%%%%%%%%%%%%%%%%%%%

\frontmatter
\maketitle

  %%%%%%%%%%%%%%%%%%%%%%%%%%%%%%%%%%%%%%%%%%%%%%%%%%%%%%%%%%%
  %   Si puo` scegliere fra scrivere tutta la tesi in un    %
  %   solo file, oppure distribuire ogni capitolo in un     %
  %   file a parte. Qui si e` scelto tenere separati i      %
  %   vari capitoli, che vengono caricati con \include      %
  %%%%%%%%%%%%%%%%%%%%%%%%%%%%%%%%%%%%%%%%%%%%%%%%%%%%%%%%%%%


	%%%%%%%%%%%%
	%	   %
	% Licenza  %
	%          %
	%%%%%%%%%%%%

\subsection*{Licenza} \label{lic}
Quest'opera, per volontà dell'autore, è rilasciata sotto la disciplina della seguente licenza: Creative Commons Public License Attribuzione - NonCommerciale - CondividiAlloStessoModo 2.5 Italia.\\
Tu sei libero: 
\begin{itemize}
\item  di riprodurre, distribuire, comunicare al pubblico, esporre in pubblico, rappresentare, eseguire e recitare quest'opera
\item di modificare quest'opera

\end{itemize}
Alle seguenti condizioni:
   \begin{itemize}
\item Attribuzione. Devi attribuire la paternità dell'opera nei modi indicati dall'autore o da chi ti ha dato l'opera in licenza. 	
\item Non commerciale. Non puoi usare quest'opera per fini commerciali.
\item Condividi allo stesso modo. Se alteri o trasformi quest'opera, o se la usi per crearne un'altra, puoi distribuire l'opera risultante solo con una licenza identica a questa.
\item Ogni volta che usi o distribuisci quest'opera, devi farlo secondo i termini di questa licenza, che va comunicata con chiarezza.
\item In ogni caso, puoi concordare col titolare dei diritti d'autore utilizzi di quest'opera non consentiti da questa licenza.

\end{itemize}

\textbf{Le utilizzazioni consentite dalla legge sul diritto d'autore e gli altri diritti non sono in alcun modo limitati da quanto sopra.}

Questo è un riassunto in linguaggio accessibile a tutti del Codice Legale (la licenza integrale) che è disponibile alla pagina web:\\
\verb|http://creativecommons.org/licenses/by-nc-sa/2.5/legalcode.it|
% \begin{figure}[htb]
% \centering
% \includegraphics[scale=0.6]{CC}
% \caption[Il logo di Creative Commons]{Il logo di Creative Commons\label{fig:creativecommons}}
% \end{figure}
\tableofcontents
\listoffigures
%\lstlistoflistings


%\lstset{language=C++}

\lstset{
%basicstyle=\small,
%numbers=none,
%numberstyle=\tiny,
%stepnumber=1,
%numbersep=2pt,
%frame=TB,
%framesep=5pt,
%  xleftmargin=0.3cm,
basicstyle=\ttfamily,
keywordstyle=\color{blue}\ttfamily,
ndkeywordstyle=\color{yellow}\ttfamily,
identifierstyle=\ttfamily,
commentstyle=\color{red}\ttfamily,
stringstyle=\color{black}\ttfamily,
%directivestyle=\color{magenta}\ttfamily
}



\chapter{Introduzione e Obiettivi}

\section{Confronto dei metodi di tracking basati sul modelli}
\subsection{Kalman Filter}
\subsection{ConDenSation}
\section{Descrizione degli algoritmi}
Inserire qui riferimenti del  Bibi\TeX.
\section{Descrizione dell' implementazione dei modelli}
Cosa si prende per varianza di uno dell'altro, come è fatto lo stato (vettore i 4 dimensioni di cui 2 posizione xy etc..)

\mainmatter
\chapter{Proprietà Intellettuale e Industriale}
cosa sono , differenze etc
\section{Diritto d'autore e Copyright}
breve excursus su tutti e due. 1 pagina e mezzo massimo.
\section{Marchi}
qui bisogna spiegarlo x bene
\section{Brevetti}
Idem, questo è il cardine di tutta la relazione. Va data la giusta definizione, confrontati le varie legge in ogni Stato, e inserendo dierse sezioni in maniera tale da aumentare il volume
	\subsection{Royalties}

\section{Analogie e Differenze}
Questa sezione piu che lunga deve essere efficace e sintetica. magari con una tabellina

\chapter{La proprietà intellettuale in campo software}
Dopo aver delineato la sfera della Proprietà Industriale e Intellettuale, si cerca di fare luce sull'attuale sistema di tutela dei diritti che si acquisicono con la scrittura di un programma per elaboratori; si cerca di cogliere in esso la giusta forma di tutela, confrontando realtà diverse come in UE e in USA. 

\section{Il sistema giuridico delle licenze}
Attualmente il software è concepito come espressione dell'intelletto umano e ovviamente ricade nella tutela della Proprietà Intellettuale. Per questo chiunque scrive un programma per elaboratore detiene il così detto copyright, cioè tutti i diritti che si sono evidenziati nel capitolo 1.\\
Principalmente il software è prodotto con enormi investimenti per uno scopo molto semplice: l'uso. Negli anni '80 e '90, imprese come la famosa Microsoft (e non solo) hanno lucrato su questa rivoluzione digitale fino a diventare multinazionali o comunque colossi dell'informatica.\\
Si è cercato fin da subito quindi di tutelare il software in maniera tale da avere la possiblità di venderlo, pur rimanendo sempre i proprietari. Come si sa su di esso si possiede il diritto d'autore (cioè la proprietà intellettuale, non la proprietà del supporto con cui viene venduto) ed è proprio su questa facoltà che è nato il contratto tra un licenziatario \footnote{colui che ne detiene il copyright} e un licenziante \footnote{qualsiasi utente dell'opera}. Questo accordo scritto prende comunemente il nome di licenza.
Nel particolare la licenza segue il modello uno a molti licenziante-licenziatario e il concetto di ``contratto di licenza'' si definisce come un atto unilaterale giuridico, originario del diritto amministrativo, con cui un soggetto concede un' autorizzazione a compiere una determinata attività.
Ovviamente l'attività con cui ci si riferisce in questo caso è quella che deriva dalla cessione di alcuni diritti come, ad esempio, il diritto all'uso: infatti con l'acquisto di un software non si compra il software in sè, inalienabile dall'autore, ma si comprano i diritti all'uso e in alcuni casi (come nel copyleft) anche il diritto alla modifica e alla copia/ridistribuzione.

\section{La situazione dell' UE sui brevetti software}

\section{I brevetti software in USA}

 \chapter{Il software libero nel sistema giuridico informatico}

Appurate le definizioni di forma di tutela della Proprietà Intellettuale come copyright, brevetto e marchio nel capitolo 1 e la storia dei brevetti applicati software nei mercati più sviluppati come USA e UE nel capitolo 2, si sfrutta i seguenti due capitolo dell'elaborato per approfondire il dibattito sulla tutela del software che oscilla tra copyrigt e brevetti.

In particolare nel capitlo 3 si approfondirà come il movimento opensource stia cercando di difendersi dalla brevettazione del software attraverso la licenza madre copyleft \textit{par excellence}: la GPL, modificata di recente per questo motivo approdando alla terza versione.
In questo trattato non si approfondiranno nè i concetti di opensource o free software nè il concetto di copyleft, che invece possono essere conosciuti rispettivamente attraverso la lettura di \cite[Compendio di libertà informatica e cultura open]{Aliprandi-compendio} e \cite[Copyleft e Opencontent]{Aliprandi-copyleft}

Il capitolo 4 invece sarà più pratico: si metteranno in luce delle conseguenze che la brevettazione software apporta sia nel software libero che in quello proprietario e commerciale; quindi si farà un esempio di brevetto, verrà spiegato e si vedrà le conseguenze portati in USA e in UE.

Adesso si va ad approfondire alla luce di quanto visto fino ad ora, come si tutela il \textit{free software} nel sistema giuridico informatico.


\section{Le licenze libere}
Breve introduzione
	\subsection{La licenza GPL}
	breve
\section{L'approccio ai brevetti della nuova licenza GPLv3}
va letta la GPL3 e spulciata per quanto riguarda i brevetti riportando magari qualche tratto saliente.
\section{Un esempio di brevetto vincolato dalla GPLv3}
idem


\chapter{Un esempio concreto: il caso MP3}
Per rendere chiara e comprensibile la trattazione sarà portata come esemplificativa uno delle questioni più famose e rilevanti nella storia della brevettazione software, sia per l'importanza dell'invenzione sotto brevetto, sia per la rilevanza economica che ha avuto la causa giudiziaria di violazione di brevetto che ha colpito una delle più importanti aziende informatiche: Microsoft Corporation.

Rilevante sarà anche l'analisi del comportamento che deve essere tenuto nelle varie regioni mondiali in virtù della valenza o meno del suddetto brevetto: per chiarire questa problematica sarà portato come esemplare il comportamento della distribuzione GNU/Linux più diffusa al momento, Ubuntu Linux\cite{ubuntu}.

\section{Cos'è l'algoritmo di compressione MP3}
La discussione di questo capitolo verte su uno degli algoritmi più importanti della storia informatica degli ultimi tempi: l'algoritmo di compressione audio \textit{MPEG-1/2 Audio Layer 3}, comunemente noto come MP3. Questo algoritmo di compressione è diventato noto per la sua capacità di ridurre drasticamente la quantità di dati richiesti per la riproduzione di un suono, mantenendo comunque una riproduzione fedele del suono originario. Nei moderni codificatori MP3 gli algoritmi più efficaci fanno di tutto per assicurare che i suoni rimossi siano quelli che non possono essere rilevati dall'orecchio umano. Questo risultato è stato ottenuto anche grazie alla scienza della psicoacustica.

Nonostante ne siano stati riconosciuti molti difetti, diversi dei quali superati anche da algoritmi successivi ed alternativi (si pensi all'algoritmo \textit{AAC MPEG-4} oppure all'\textit{Ogg Vorbis}) il formato .mp3 (classico dei files compressi con tale algoritmo) risulta ancora il più diffuso in campo musicale, e ciò spiega la portata economica che può comportare l'eventuale copertura brevettuale sull'invenzione.
\section{Il brevetto sull'MP3}\label{mp3-patent}
La Thomson Consumer Electronics è la proprietaria principale del brevetto di MPEG-1/2 Layer 3 in U.S.A. e Giappone, e ha raccolto in un apposito sito (\textit{http://www.mp3licensing.com/}) tutte i brevetti relativi all'MP3 che detiene (svariati validi anche in UE), e una riepilogativa tabella delle royalties che le aziende devono pagare per utilizzare codificatori e decodificatori di MP3.
\begin{figure}[b]
	\begin{center}
		\includegraphics[scale=0.75]{figure/mp3.jpg}
	\end{center}
	\caption{\textit{Il logo del sito Thomson sui brevetti MP3}}
\end{figure}
\subsection{Ricerca del brevetto}

\subsection{Termini del brevetto}

\subsection{Aree di valenza e royalties}

\section{Il delicato rapporto tra Microsoft ed il formato MP3}
Scendendo nelle notizie di attualità è comune trovare, in campo tecnico/ingegneristico, notizie di violazioni di brevetto, di violazione di proprietà intellettuali e simili. Un po' meno raro è trovare eventi che vedano implicati i brevetti software, specialmente in Europa, dove, come abbiamo visto, sono quasi impossibili da ottenere. 

\`E normale quindi che faccia scalpore quando un tribunale emette una sentenza di violazione di brevetto informatico contro una azienda; è ancora più normale che l'interesse salga a livelli inauditi se l'azienda coinvolta è la più fiorente in campo informatico, e viene condannata ad una pena pecuniaria pari al fatturato di una decina di anni di una azienda normale. Stiamo parlando della Microsoft, e della \textit{querelle} giudiziaria che l'ha vista protagonista con Alcatel-Lucent.

La questione è spinosa in quanto non vede solo motivazioni giuridiche tra i proprietari originari del brevetto ed il presunto violatore, ma vede di fronte al presunto violatore delle enormi aziende che hanno inglobato in percentuali diverse le originali proprietarie del brevetto, lasciando innescare quindi procedure economico/giudiziarie dalla portata enorme.

Nel caso di MP3, come è stato detto nella sezione \ref{mp3-patent}, il principale detentore della proprietà brevettuale è Thomson Consumer Electronics, ma non è affatto ne' l'unica ne' l'originaria proprietaria. Gli algoritmi di base di MP3 sono stati sviluppati originariamente in collaborazione tra il Fraunhofer Institute e gli ex-Bell Laboratories. Il primo gruppo a rilasciare un encoder fu il Fraunhofer Institute nel 1994, e Microsoft ha sempre sostenuto di aver ottenuto in licenza la tecnologia proprio da quest'ultimo, pagandola ben 16 milioni di dollari ed integrandola nei sistemi operativi Windows attraverso i codec e il lettore software Windows Media Player.

Thomson è di fatto la società che al momento controlla il Fraunhofer Institute, mentre Alcatel-Lucent al momento detiene la proprietà dei Bell Laboratories.

Proprio Alcatel-Lucent nel 2003 ha trascinato in tribunale i produttori di PC Dell e Gateway per l'utilizzo illegittimo dei suoi brevetti. Microsoft, in accordo con i patti di indennizzo stretti con le due società, ha offerto loro protezione legale ed ha ottenuto come contropartita la denuncia di Alcatel per la violazione degli accordi di sfruttamento dei brevetti sulla console Xbox 360. Le due aziende avevano stretto un'intesa sulla prima Xbox ma Alcatel-Lucent ha sostenuto davanti al giudice - e ha infine ottenuto una sentenza a proprio favore - che l'accordo non comprendeva la nuova versione; in tutto la disputa riguardava ben quindici violazioni di brevetto, e dopo il rigetto delle prime due accuse, nel Febbraio del 2007 è arrivata la notizia di una sconfitta giuridica per la Microsoft, per la violazione appunto del brevetto riguardante MP3. La sanzione prevedeva una multa per più di un miliardo e mezzo di dollari, valutati i benefici sfruttati abusivamente da Microsoft nel proprio sistema, valutata la diffusione del formato MP3 e la diffusione del sistema Microsoft stesso.

Il motivo principale della diatriba è strettamente legato alle royalties\footnote{Con il termine royalty si indica il pagamento di un compenso al titolare di un brevetto o una proprietà intellettuale, con lo scopo di poter sfruttare quel bene per fini commerciali.} che le aziende devono pagare per poter utilizzare il formato MP3. Come si può vedere nell'elenco pubblico disponibile nel sito Thomson già citato in precedenza, Microsoft risulta considerata tra le aziende autorizzate all'utilizzo della tecnologia; Alcatel, cercando di sfruttare altri processi già aperti contro la casa di Redmond ha tentato di avvalersi del presunto diritto di riscuotere ulteriori royalties sul formato, in virtù dell'acquisizione dei Bell Laboratories.

La questione, ancora non definitivamente sciolta in quanto è ancora possibile un ulteriore grado di giudizio, ha visto la sentenza in appello ribaltare, e di fatto annullare, la sentenza contro Microsoft, riconoscendo sufficiente il pagamento del brevetto presso uno dei proprietari legittimi.

 
\section{Il caso Ubuntu: come utilizzare gli MP3 senza violare il brevetto}

% \section{Cosa si può fare/non fare in USA in questa circostanza}
% \section{Cosa si può fare/non fare in UE in questa circostanza}

%\include{capitolo5}
 \chapter{Conclusioni}

Con questo elaborato si è voluto differenziare nel primo capitolo i concetti di Proprietà Intellettuale e Industriale, che pur rimanendo molto simili, nascono da un sostrato abbastanza diverso, anche se rispettivamente la prima copre una serie di diritti maggiore della seconda, come riportato nella visione insiemistica in figura \ref{fig:PI} . I tipi di tutela quindi che derivano da entrambi sono leggermente diversi in quanto il primo si attaglia al diritto d'autore in generale e l'altro alla tutela dell'invenzione come prodotto industriale. Fatto questo si è passati a definire brevemente il concetto di copyright, dilungandosi invece leggermente nella descrizione dei brevetti e marchi, forme di tutela della Proprietà Industriale, in quanto più coerenti con la traccia della trattazione.

Il capitolo secondo si contraddistingue perché, dopo aver data la definizione di brevetto, entra nel merito di tutta la trattazione: la tutela del software attraverso lo strumento del brevetto. Su questo concetto emergono tutt'oggi pareri contrastanti, per questo motivo la trattazione si è fatta molto interessante. Si è percorso un breve excursus sulla tutela giuridica del software secondo copyright, successivamente si è riassunto la storia dei brevetti software in UE e in USA, dandone una visione dall'alto e riportando i casi giuridici principali di ogni paese.

La trattazione poi si è spostata su tempi più recenti e ambiti tecnici. In particolare nel capitolo terzo si è osservato il graffiante punto di vista della Free Software Foundation nei confronti dei software patents, legati anche ad altre tecnologie recenti di restrizione dei diritti degli utenti. Si è quindi analizzato i nuovi articoli che compongono la nuova licenza copyleft: la GNU General Public License versione 3.
Infine si è dato un esempio degli effetti provati da questa, nella distribuzione del software gplv3 con software house che hanno sottoscritto patents agreement con altre: lampante in questo è il caso Novell-Microsoft e il passaggio di licenza del software di interoperabilità tra reti SMB.

Ricordando di dare una veste pratica oltre che legale, l'ultimo capitolo prende come esempio il brevetto del formato di compressione audio per antonomasia, cioè l'mp3: il brevetto non è unico, ma sono una serie di brevetti detenuti da molte società; è stato quindi effettuato un lungo lavoro per districarsi nella rete brevettuale che si è formata. Prendendo questo caso, si vuole mostrare quali sono gli effetti della brevettazione, della difficoltà legata alla eterogeneità delle varie legislazioni e degli effetti apportate al free software o a software house commerciali.

Dalla trattazione è quindi emerso che il software è giusto che venga tutelato dal diritto d'autore; è anche vero d'altro canto che non può essere concepito solo come una mera composizione logico-intellettuale, visto l'enorme commercio del mercato del software. E quindi in questo senso è giusto affermare che anche il software, pur rimanendo un bene immateriale diverso dall'hardware, è entrato nella schiera dei prodotti industriali a tutti gli effetti. Appurato questo è giusto chiedersi se è lecito lasciare il software sotto copyright o addirittura applicare la nozione di brevetto, concepita per prodotti fisici o processi industriali, anche agli algoritmi implementati. Oppure invece notando l'uso che poi ne viene fatto dalle software house, potrebbe essere possibile negare l'applicazione del brevetto al software, in maniera congrua con quanto affermato dalla Free Software Foundation nella GPLv3, magari trovando una via alternativa tra la tutela intellettuale troppo generica e la tutela industriale troppo legata al monopolio.

Al termine del lavoro è necessario ringraziare tutte le fonti non bibliografiche che hanno reso possibile la scrittura di questa tesina. Mi riferisco in modo particolare ai fatti di attualità, sengalati con puntualità e dettaglio da Punto Informatico (\textit{http://punto-informatico.it/}), Computer World Online (\textit{http://www.cwi.it/}). Un ringraziamento particolare anche a Free Software Foundation Europe (\textit{http://www.italy.fsfeurope.org}), sito ricco di riferimenti e dettagli cronologici molto interessanti.


\appendix

%\include{appendici}

\backmatter
\bibliographystyle{IEEEtran}
\bibliography{bibliografia}
%\include{biblio}

%\printindex % se si fa l'indice analitico.

\end{document}
