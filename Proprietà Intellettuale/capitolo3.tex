 \chapter{Il software libero nel sistema giuridico informatico}

Appurate le definizioni di forma di tutela della Proprietà Intellettuale come copyright, brevetto e marchio nel capitolo 1 e la storia dei brevetti applicati al software nei mercati più sviluppati come USA e UE nel capitolo 2, si sfrutta i seguenti due capitolo dell'elaborato per approfondire il dibattito sulla tutela del software che oscilla tra copyrigt e brevetti.

In particolare nel capitolo 3 si approfondirà come il movimento opensource stia cercando di difendersi dalla brevettazione del software attraverso la licenza madre copyleft \textit{par excellence}: la GPL; la quale è stata tra l'altro modificata di recente per questo motivo, approdando alla terza versione.
In questo trattato si danno per scontati i concetti di opensource o free software e il concetto di copyleft: essi comunque possono essere conosciuti rispettivamente attraverso la lettura di \cite[Compendio di libertà informatica e cultura open]{Aliprandi-compendio} e \cite[Copyleft e Opencontent]{Aliprandi-copyleft}.

Il capitolo 4 invece sarà più pratico: si metteranno in luce delle conseguenze che la brevettazione software apporta sia nel software libero che in quello proprietario; quindi si farà un esempio di brevetto,spiegandone le specifiche e si vedrà le conseguenze esportate dal brevetto nello sviluppo Americano e Europeo.

Adesso si inizia ad approfondire, alla luce di quanto visto fino ad ora, come si tutela il \textit{free software} nel sistema giuridico informatico.


\section{Le licenze libere}

Nella sezione \ref{sec:sistema-licenze} si è affrontato la gestione delle licenze e si è sottolineato la natura di contratto ad adesione per le licenze copyleft o libere.

In generale una licenza software si dice libera quando è rilasciata attraverso la gestione dei diritti tipica del copyleft: in questo caso il software è quasi sempre rilasciato gratuitamente e l'utente che accetta la licenza possiede non solo il diritto di uso, ma anche quello di modifica e copia. L'unico dovere dell'utente risulta quello di distribuzione del software, dopo eventuali modifiche, sotto la stessa identica licenza a cui era sottoposto il programma originale. Questo aspetto delle licenze copyeft è detto viralità delle licenze: l'aspetto virale sta nel fatto di continuare a propagarsi sul software free sviluppato, tutelandone quindi la sua natura \textit{libera} apputo.

\`E bene ricordare che il concetto di copyleft non si attaglia solo al prodotto di software ma può essere applicato a molti prodotti come musica, blog informativi, enciclopedie, libri o ebook: le licenze libere non necessariamente devono riguardare solo programmi per elaboratore. Se si pensa  all' esempio di Wikipedia, l'enciclopedia libera,  esistono licenze libere per la documentazione come la GNU FDL; oppure se si pensa alla pubblicazione di libri e materiale informativo esistono le cosìdette licenze opencontet come quelle pubblicate dal gruppo Creative Commons.
Fondamentalmente la causa scatenante della gestione alternativa del copyleft, ma che si basa sul copyright, ha preso campo con l'avvento di Internet e la condivisione di migliaia di documenti online, provocando una vera rivoluzione per quanto riguarda la tematica del diritto informatico.


\subsection{La licenza GPL}

Dopo aver accenato alle licenze libere in generale e visto che tale gestione non si attaglia solo al software, si prende in analisi la madre di tutte le licenze di tipo copyleft, cioè la GNU General Public License.

La GNU General Public License è una licenza per software libero. Anzi è possibile dire che vi è software libero quando questo è rilasciato sotto licenza GPL.

La GNU GPL è stata scritta da Richard Stallman e Eben Moglen nel 1989 (versione 1.0), rivista nel 1991 (versione 2.0) e il 29 Giugno del 2007 è uscita l'aspettata versione 3.0. Si approfondirà il dibattituo sulla versione 3 nelle sezioni successive, adesso ci si soffermerà sulla trattazione in generale della licenza fino alla versione 2, attualmete la più utilizzata del mercato del free software.

Contrapponendosi alle licenze per software proprietario, la GNU GPL permette all'utente libertà di utilizzo, copia, modifica e distribuzione; a partire dalla sua creazione è diventata una delle licenze per software libero più usate. Il testo della GNU GPL è disponibile per chiunque riceva una copia di un software coperto da questa licenza. I licenziatari/utenti che accettano le sue condizioni hanno la possibilità di modificare il software, di copiarlo e ridistribuirlo con o senza modifiche, sia gratuitamente sia a pagamento. Quest'ultimo punto distingue la GNU GPL dalle licenze che proibiscono la ridistribuzione commerciale.\footnote{come quella utilizzata per questo'opera}.

Se l'utente distribuisce copie del software, deve rendere disponibile il codice sorgente a ogni acquirente, incluse tutte le modifiche eventualmente effettuate; nella pratica, i programmi sotto GNU GPL vengono spesso distribuiti allegando il loro codice sorgente, anche se la licenza non lo richiede. Ci sono casi in cui viene distribuito solo il codice sorgente e in quel caso è l'utente che ha il compito di compilarlo, ricavando il formato eseguibile per l'uso.
L'utente ha il dovere di rendere disponibile il codice sorgente solo alle persone che hanno ricevuto da lui il formato eseguibile. Questo significa, ad esempio, che è possibile creare versioni private di un software sotto GNU GPL, a patto che tale versione non venga distribuita a qualcun altro. Questo accade quando l'utente crea delle modifiche private al software, ma non lo distribuisce: in questo caso non è tenuto a rendere pubbliche le modifiche.

Da sottolineare nella licenza anche la clausola di non garanzia, reperibile all'art.11. Dato che il software è protetto da copyright, l'utente non ha altro diritto di modifica o ridistribuzione al di fuori dalle condizioni di copyleft. In ogni caso, l'utente deve accettare i termini della GNU GPL se desidera esercitare diritti normalmente non contemplati dalla legge sul copyright, come la ridistribuzione. Se qualcuno distribuisce un software (in particolare, versioni modificate) senza rendere disponibile il codice sorgente o violando in altro modo la licenza, può essere denunciato dall'autore originale secondo le stesse leggi sul copyright.

In questo senso la GPL risulta un intelligente cavillo legale e per questo è stata descritta come un \textit{``copyright hack''}, che riesce a mantenere le famose quattro libertà rivendicate dal suo creatore R.~Stallman quali:
\begin{itemize}
\item la libertà di usare a propria discrezione (libertà 0)
\item la libertà di copiare e condividere con altri (libertà 1)
\item la libertà di modificare, studiare ed elaborare (libertà 2)
\item la libertà di ridistribuire i cambiamenti e i lavori derivati a patto di mantenre la solita licenza (libertà 3)
\end{itemize}


\section{L'approccio ai brevetti della nuova licenza GPLv3}
Il 29 Giugno 2007 è stata rilasciata a Boston (Massachusetts) la terza versione della GPL, creata dalla  Free Software Foundation (FSF).
Per introdurre la licenza si riporta uno spezzone del discorso del creatore R.~Stallman:

%``Tivoization and Treacherous (aka, “Trusted”) Computing are schemes to prevent users from utilizing modified or alternate software. The former simply blocks modified software from running; the latter enables web sites to refuse to talk to modified software. Both are typically used to impose malicious features such as Digital Restrictions Management (DRM). GPL version 3 does not restrict the features of a program; in particular, it does not prohibit DRM. However, it prohibits the use of tivoization and Treacherous Computing to stop users from changing the software. Thus, they are free to remove whatever features they may dislike. ''

``Tivoizzazione e Trusted Computig sono - afferma Stallman - dei sistemi per prevenire l'utilizzo di versione modificate di un software da parte dell'utente. Questo si realizza semplicemente evitando di eseguire il software modificato, impedendone la comunicazione con l'hardware. Entrambi i metodi sono imposti per inserire caratteristiche similie al DRM, ma più nocive per l'utente. La GPL3 non vuole restringe le caratteristiche di un programma; in particolare non proibisce il DRM. Comunque, proibisce l'uso di processo di Tivoizzazioni e TC che impediscono all'utente di modificare il software. In questo modo gli utente sono liberi di rimuovere le caratteristiche di un programma che potrebbero non piacergli.''

Da quanto afferma Stallman si intravedono subito le nuove tematiche relative all' DRM, all'aggiramento di TiVo alla GPLv2 e al Trusted Computing. \`E giusto soffermarsi sulla questione TiVo in quanto è uno delle cause della nascita della GPLv3: infatti il software usato nei dispositivi di videoregistrazione TiVo era free software quindi sotto licenza GPLv2; però le modifiche che l'utente apportava al software venivano annullate dai controlli hardware effettuati dalla macchina. A questo concetto, quello di appropriarsi del software libero e attraverso mezzi hardware togliere la famosa libertà 3 all'utente, si aggancia anche il discorso del Trusted Computing. La licenza, anche se non citato da Stallman, tratta anche i cosìdetti ``software patents''; infatti essa è stata rilasciata in concomitanza con gli accordi tra Microsoft e Novell, per legalizzare i brevetti che il kernel Linuxe gli applicativi GNU avrebbero infranto nei confronti della casa di Redmond.

\begin{figure}[b]
	\begin{center}
		\includegraphics[scale=0.5]{figure/gplv3.png}
	\end{center}
	\caption{Il logo della GPLv3}
\end{figure}

\subsection{Caratteristiche}
La prima caratteristica della licenza GPLv3 da mettere in luce è che essa non è retrocompatibile con la GPLv3. Quindi in questo caso è impossibile combinare del codice rilasciato sotto la versione 3 con quello sotto versione 2. \`E invece possibile aggiornare alla terza versione un software rilascito sotto versione 2. Dall'altra parte invece, con la fusione di due librerie o programmi GPLv2 e GPLv3 si intendo solo il collegamento che viene effettuato da un compilatore quando si vuole fondere due programmi in uno solo; in questo senso non recrimina l'uso di diversi programmi GPLv2 e GPLv3 su un sistema operativo: essi possono coesistere.
Un'altro aspetto nuova ma già citato è la cosidetta Tivoizzazione: apparecchi che contegono gpled software, che però in pratica non danno la possibilità di ottenere dei benefici nella modifica del free software, in quanto l'apparecchio viene automaticamente spento via hardware una volta riconosciuto il software modificato. La ragione di questo procedimento è di sfruttare la libertà e la gratuiticità del free software senza dare indietro però i benefici di tale libertà all'utente finale.





%CONTINUARE IL DISCORSO DA http://www.gnu.org/licenses/rms-why-gplv3.html inseriendo spezzone di licenza da http://katolaz.homeunix.net/gplv3/






\section{Un esempio di brevetto vincolato dalla GPLv3}
idem

