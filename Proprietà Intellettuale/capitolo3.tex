 \chapter{Il software libero nel sistema giuridico informatico}

Appurate le definizioni di forma di tutela della Proprietà Intellettuale come copyright, brevetto e marchio nel capitolo 1 e la storia dei brevetti applicati al software nei mercati più sviluppati come USA e UE nel capitolo 2, si sfrutta i seguenti due capitolo dell'elaborato per approfondire il dibattito sulla tutela del software che oscilla tra copyright e brevetti.

In particolare nel capitolo 3 si approfondirà come il movimento opensource stia cercando di difendersi dalla brevettazione del software attraverso la licenza madre copyleft \textit{par excellence}: la GPL; la quale è stata tra l'altro modificata di recente per questo motivo, approdando alla terza versione.
In questo trattato si danno per scontati i concetti di \textit{opensource} o \textit{free software} e il concetto di copyleft: essi comunque possono essere conosciuti rispettivamente attraverso la lettura di \cite[Compendio di libertà informatica e cultura open]{Aliprandi-compendio} e \cite[Copyleft e Opencontent]{Aliprandi-copyleft}.

Il capitolo 4 invece sarà più pratico: si metteranno in luce delle conseguenze che la brevettazione software apporta sia nel software libero che in quello proprietario; quindi si farà un esempio di brevetto,i metodi di ricerca spiegandone le specifiche e si vedrà le conseguenze esportate dal brevetto nello sviluppo Americano e Europeo.

Adesso si inizia ad approfondire, alla luce di quanto visto fino ad ora, come si tutela il \textit{free software} nel sistema giuridico informatico.


\section{Le licenze libere}

Nella sezione \ref{sec:sistema-licenze} si è affrontato la gestione delle licenze e si è sottolineato la natura di contratto ad adesione per le licenze copyleft o libere.

In generale una licenza software si dice libera quando è rilasciata attraverso la gestione dei diritti tipica del copyleft: in questo caso il software è quasi sempre rilasciato gratuitamente (condizione comunque non necessaria) e l'utente che accetta la licenza possiede non solo il diritto di uso, ma anche quello di modifica e copia. L'unico dovere dell'utente risulta quello di distribuzione del software, dopo eventuali modifiche, sotto la stessa identica licenza a cui era sottoposto il programma originale. Questo aspetto delle licenze copyleft è detto viralità delle licenze: l'aspetto virale sta nel fatto di continuare a propagarsi sul software free sviluppato, tutelandone quindi la sua natura \textit{libera} appunto.

\`E bene ricordare che il concetto di copyleft non si attaglia solo al prodotto di software ma può essere applicato a molti prodotti come musica, blog informativi, enciclopedie, libri o ebook: le licenze libere non necessariamente devono riguardare solo programmi per elaboratore. Se si pensa  all' esempio di Wikipedia, l'enciclopedia libera,  esistono licenze libere per la documentazione come la GNU FDL; oppure se si pensa alla pubblicazione di libri e materiale informativo esistono le cosiddette licenze opencontet come quelle pubblicate dal gruppo Creative Commons\footnote{Questo stesso documento è coperto da una licenza Creative Commons}.
Fondamentalmente la causa scatenante della gestione alternativa del copyleft, ma che si basa sul copyright, ha preso campo con l'avvento di Internet e la condivisione di migliaia di documenti online, provocando una vera rivoluzione per quanto riguarda la tematica del diritto informatico.

Poiché non si vuole approfondire la nascita sociale e le basi giuridiche del copyleft, nella prossima sezione si riassumerà le caratteristiche salienti della licenza GPL fino alla versione 2.


\subsection{La licenza GPL}

Dopo aver accennato alle licenze libere in generale e visto che tale gestione non si attaglia solo al software, si prende in analisi la madre di tutte le licenze di tipo copyleft, cioè la GNU General Public License. Si useranno termini tecnici dell'Ingegneria e della produzione del Software, dando per scontato l'uso di termini come ``codice sorgente'' o ``eseguibile'' ad esempio, senza approfondire i concetti tecnici in questa sede.

La GNU General Public License è una licenza per software libero. Anzi è possibile dire che vi è software libero quando questo è rilasciato sotto licenza GPL.

La GNU GPL è stata scritta da Richard Stallman e Eben Moglen nel 1989 (versione 1.0), rivista nel 1991 (versione 2.0) e il 29 Giugno del 2007 è uscita l'attesa versione 3.0. Si approfondirà il dibattito sulla versione 3 nelle sezioni successive, adesso ci si soffermerà sulla trattazione in generale della licenza fino alla versione 2, attualmente la più utilizzata del mercato del free software.

Contrapponendosi alle licenze per software proprietario, la GNU GPL permette all'utente libertà di utilizzo, copia, modifica e distribuzione; a partire dalla sua creazione è diventata una delle licenze per software libero più usate. Il testo della GNU GPL è disponibile per chiunque riceva una copia di un software coperto da questa licenza. I licenziatari/utenti che accettano le sue condizioni hanno la possibilità di modificare il software, di copiarlo e ridistribuirlo con o senza modifiche, sia gratuitamente sia a pagamento. Quest'ultimo punto distingue la GNU GPL dalle licenze che proibiscono la ridistribuzione commerciale\footnote{come quella utilizzata per quest'opera}.

Se l'utente distribuisce copie del software, deve rendere disponibile il codice sorgente a ogni acquirente, incluse tutte le modifiche eventualmente effettuate; nella pratica, i programmi sotto GNU GPL vengono spesso distribuiti allegando il loro codice sorgente, anche se la licenza non lo richiede. Ci sono casi in cui viene distribuito solo il codice sorgente e in quel caso è l'utente che ha il compito di compilarlo, ricavando il formato eseguibile per l'uso.
L'utente ha il dovere di rendere disponibile il codice sorgente solo alle persone che hanno ricevuto da lui il formato eseguibile. Questo significa, ad esempio, che è possibile creare versioni private di un software sotto GNU GPL, a patto che tale versione non venga distribuita a qualcun altro. Questo accade quando l'utente crea delle modifiche private al software, ma non lo distribuisce: in questo caso non è tenuto a rendere pubbliche le modifiche.

Da sottolineare nella licenza anche la clausola di non garanzia, reperibile all'art.11. Dato che il software è protetto da copyright, l'utente non ha altro diritto di modifica o ridistribuzione al di fuori dalle condizioni di copyleft. In ogni caso, l'utente deve accettare i termini della GNU GPL se desidera esercitare diritti normalmente non contemplati dalla legge sul copyright, come la ridistribuzione. Se qualcuno distribuisce un software (in particolare, versioni modificate) senza rendere disponibile il codice sorgente o violando in altro modo la licenza, può essere denunciato dall'autore originale secondo le stesse leggi sul copyright.

In questo senso la GPL risulta un intelligente cavillo legale e per questo è stata descritta come un \textit{``copyright hack''}, che riesce a mantenere le famose quattro libertà rivendicate dal suo creatore R.~Stallman quali:
\begin{itemize}
\item la libertà di usare a propria discrezione (libertà 0)
\item la libertà di copiare e condividere con altri (libertà 1)
\item la libertà di modificare, studiare ed elaborare (libertà 2)
\item la libertà di ridistribuire i cambiamenti e i lavori derivati a patto di mantenere la solita licenza (libertà 3)
\end{itemize}


\section{La GPLv3 e il rapporto con i brevetti software}
Il 29 Giugno 2007 è stata rilasciata a Boston (Massachusetts) la terza versione della GPL, creata dalla  Free Software Foundation (FSF).
Per introdurre la licenza si riporta uno spezzone del discorso del creatore R.~Stallman:

%``Tivoization and Treacherous (aka, “Trusted”) Computing are schemes to prevent users from utilizing modified or alternate software. The former simply blocks modified software from running; the latter enables web sites to refuse to talk to modified software. Both are typically used to impose malicious features such as Digital Restrictions Management (DRM). GPL version 3 does not restrict the features of a program; in particular, it does not prohibit DRM. However, it prohibits the use of tivoization and Treacherous Computing to stop users from changing the software. Thus, they are free to remove whatever features they may dislike. ''

``\textit{Tivoizzazione} e \textit{Trusted Computig} \footnote{entrambi i termini saranno spiegati in seguito} sono - afferma Stallman - dei sistemi per prevenire l'utilizzo di versione modificate di un software da parte dell'utente. Questo si realizza semplicemente evitando di eseguire il software modificato, impedendone la comunicazione con l'hardware. Entrambi i metodi sono imposti per inserire caratteristiche simile al DRM (Digitals Rights Management) \footnote{Il significato di DRM sta nella gestione dei diritti d'autore automatizzata nell'era digitale dello scambio sfrenato di file protetti da copyright. Che in linea di principio non è combattuto dalla FSF. Quello a cui si oppone la FSF è quello che loro chiamano maliziosamente con la solita sigla DRM, con il significato di Digital \textbf{Restrictions} Management: tutti gli apparati hardware che negano la libertà di pieno utilizzo all'utente.}, ma più nocive per l'utente. La GPL3 non vuole restringe le caratteristiche di un programma; in particolare non proibisce il DRM. Comunque, proibisce l'uso di processo di Tivoizzazioni e TC che impediscono all'utente di modificare il software. In questo modo \textit{[cioè utilizzando la GPLV3]} gli utente sono liberi di rimuovere le caratteristiche di un programma che potrebbero non piacergli.''

Da quanto afferma Stallman si intravedono subito le nuove tematiche relative al DRM, al problema dell'aggiramento della società TiVo alla GPLv2 e al Trusted Computing. \`E giusto soffermarsi sulla questione TiVo in quanto è uno delle cause della nascita della GPLv3 insieme ai brevetti software: il software usato nei dispositivi di videoregistrazione digitale TiVo era free software quindi sotto licenza GPLv2, in particolare veniva utilizzata una versione modificata del kernel linux 2.4 \footnote{il cui codice sorgente può essere reperito su http://dynamic.tivo.com/linux/ come prevede giustamente la GPLv2 }; l'elusione alla licenza versione 2 consisteva nel fatto che le modifiche che l'utente apportava al software venivano annullate dai controlli hardware effettuati dalla macchina. A questo concetto, cioè quello di appropriarsi del software libero e attraverso mezzi hardware togliere la famosa libertà 2 all'utente, si aggancia anche il discorso del Trusted Computing. La licenza, anche se non citato da Stallman, tratta anche i cosiddetti ``software patents''; infatti essa è stata rilasciata in concomitanza con i recenti accordi legali tra Microsoft e Novell, per legalizzare i brevetti che il kernel Linux e la suite di applicativi GNU avrebbero infranto nei confronti della casa di Redmond.

\begin{figure}[b]
	\begin{center}
		\includegraphics[scale=0.15]{figure/gplv3.png}
	\end{center}
	\caption{\textit{Il logo della GPLv3}}
\end{figure}

Questo in sintesi è l'\textit{humus}, cioè la terra fertile, su cui sono nate le nuove forme di difesa di R.~Stallman, brillantemente messe in pratica dai legali della Free Software Fondation. La GPLv3 è stata proposta nell'ottica di vedere come si pone il mondo del software libero e opensource nei confroni dei software patents, ricollegandosi ai capitoli precedenti.

Si affronteranno ora gli articoli della licenza versione 3 di maggior rilievo nella sezione \ref{sec:analisi-gpl3}.

\subsection{Analisi} \label{sec:analisi-gpl3}
Prima di tutto è bene citare dove poter reperire il testo della licenza sotto vari formati liberi come HTML, PDF, TXT. Esso infatti può essere reperito tramite web nel sito della GNU all'indirizzo \textit{http://www.gnu.org/licenses/gpl-3.0.html}. La traduzione italiana non è ufficialmente validata da FSF, ma comunque è attendibile e di grande aiuto. Si può reperire all'indirizzo \textit{http://katolaz.homeunix.net/gplv3/}.

Si va quindi ad osservare le caratteristiche saliente che la diversificano dalla versione 2 e che trattano l'argomento dei brevetti vincolati al software, ricordando che risulta una licenza che cerca di vietare i privilegi ottenuti tramite i brevetti software.

La prima caratteristica della licenza GPLv3 da mettere in luce è che essa non è compatibile con la GPLv2. Quindi in questo caso è impossibile combinare del codice rilasciato sotto la versione 3 con quello sotto versione 2. Dove con il termine \textit{combinare} si intende il termine tecnico di fusione software (fase di linking del compilatore) tra programmi e/o librerie.  \`E invece possibile aggiornare alla terza versione un software rilasciato sotto versione 2. Poiché l' incompatibilità riguarda solo la combinazione di software, in questo senso non recrimina l'uso di diversi programmi GPLv2 e GPLv3 su un sistema operativo: essi possono coesistere.

Analizzando nel dettaglio la prima parte della licenza, si nota come il preambolo, l’introduzione alla licenza vera e propria, ha subito modifiche puramente formali e cosmetiche, a parte l’inclusione di un paragrafo sulla contrarietà ai DRM e un chiarimento sul rapporto fra GPL e brevetti.
Più sostanziale, invece, la modifiche successiva all'art. 0, intitolato “Definizioni”, sul quale poggia l’intero impianto della licenza. Vale la pena di osservare che è vincolata alla GPLv3 qualsiasi forma di circolazione del software (riassunta nella licenza con il termine \textit{propagate}), mentre la versione precedente si occupava solo di \textit{copia} e \textit{ridistribuzione}. Come in molti casi questa è una modifica puramente formale che serve a ridurre le ambiguità del testo piuttosto che ad aggiungere qualcosa di nuova.

Spostandosi invece su temi più caldi si evidenziano elementi di novità invece, dall’art. 2 (Basic permission) e dall’art. 3 (Digital Restrictions Management) della licenza.
Il primo introduce prime limitazioni al diritto di uso del software da parte dell’utente finale in rapporto al rischio di provocare cause per violazione di brevetto. Gli articoli che riprendono tale argomento sono anche l'art. 10 e 11.
Il secondo vieta la distribuzione di software che violano la privacy degli utenti e limitano, in qualsiasi modo, l’esercizio dei diritti attribuiti con la GPL. Proprio come avvenuto per quanto riguarda la questione TiVo.
Riportiamo qui a seguito l'articolo 3:\\


\begin{scriptsize}

\textbf{Art. 3 - Protezione dei diritti legali degli utenti dalle leggi anti-elusione.}

Nessun programma protetto da questa Licenza può essere considerato parte di una misura tecnologica di restrizione che sottosta ad alcuna delle leggi che soddisfano l'articolo 11 del "WIPO copyright treaty" adottato il 20 Dicembre 1996, o a simili leggi che proibiscono o limitano l'elusione di tali misure tecnologiche di restrizione.
\textbf{Quando distribuisci un programma coperto da questa Licenza, rifiuti tutti i poteri legali atti a proibire l'elusione di misure tecnologiche di restrizione} ammesso che tale elusione sia effettuata nell'esercizio dei diritti garantiti da questa Licenza riguardo al programma coperto da questa Licenza, e rinunci all'intenzione di limitare l'operatività o la modifica del programma per far valere, contro i diritti degli utenti del programma, diritti legali tuoi o di terze parti che impediscano l'elusione di misure tecnologiche di restrizione.\\
\end{scriptsize}


Questo articolo ha una importanza centrale perché si riflette non solo sulla progettazione di un software, ma anche sul modello di business che dovrebbe sostenere la commercializzazione dei servizi a esso collegati. L'articolo riguarda molto di più l'uso della licenza da parte del distributore del programma, piuttosto che colui che la sottoscrive cioè l'utente. Infatti il distributore usando questa licenza rifiuta tutti i diritti derivati dal tratto WIPO sulla possibilità di proibire l'elusione a misure di restrizione di accesso sul software di tipo DRM nella veste coniata dalla FSF di Digital Restrictions Management. \`E quindi con questo articolo che la FSF si tutela da future eventuali \textit{Tivoization}. Comunque l'articolo risulta abbastanza generico non specificando la forma benigna o maligna di queste tecnologie, a volte incluse per togliere libertà all'utente, a volta atte ad avere uno stretto controllo sul software per aggiornarlo in maniera remota. \`E evidente che non sempre si potrà rispettarne il dettato: è vietata qualsiasi invasione della privacy e quindi alcuni modelli di business, anche se gestiti in buona fede, saranno comunque in violazione di licenza.

Questo aspetto quindi tocca il nuovo ma già citato argomento della cosiddetta Tivoizzazione: apparecchi, contenenti \textit{gpled software}, che non danno la possibilità di ottenere dei benefici nella modifica del free software, in quanto l'apparecchio è automaticamente spento via hardware una volta riconosciuto del software non originale. La ragione di questo procedimento è di sfruttare la libertà e la gratuiticità del free software senza dare indietro però i benefici di tale libertà all'utente finale. L'articolo 3 della GPLv3 tenta di arginare questi tipi di comportamenti.

L'articolo 9 invece recita nel seguente modo ed ha una valenza legale molto pesante soprattutto perché può essere differentemente interpretato a seconda dei paesi in cui la licenza viene usata.\\


\begin{scriptsize}\textbf{Art. 9 - L'ottenimento di copie non richiede l'accettazione della Licenza}
Non sei obbligato ad accettare i termini di questa Licenza al solo fine di ottenere o eseguire una copia del Programma. Similmente, propagazioni collaterali di un Programma coperto da questa Licenza che occorrono come semplice conseguenza dell'utilizzo di trasmissioni peer-to-peer per \textbf{la ricezione di una copia non richiedono l'accettazione della Licenza}. In ogni caso, solo e soltanto questa Licenza ti garantiscono il permesso di propagare e modificare qualunque programma coperto da questa Licenza. Queste azioni violano le leggi sul copyright nel caso in cui tu non accetti questa Licenza. Pertanto, modificando o propagando un programma coperto da questa Licenza, indichi implicitamente la tua accettazione della Licenza.\\

\end{scriptsize}

Per quanto riguarda l'Italia, si può concludere che, dal punto di vista del diritto italiano, la GPL3 sia a tutti gli effetti un contratto per adesione (tipo quelli di banche e assicurazioni) come riportato nella precedente sezione di Licenze Libere.

I brevetti software vengono spesso citati nella licenza soprattutto quando si parla di rilasciare del software che implementa dei brevetti, ma a differenza della GPLv2, adesso vi è un articolo esclusivamente dedicato, il numero 11 che commenteremo in seguito. Riportiamo dei passi salienti presi da vari articoli che trattano la distribuzione di codice sotto brevetto; in particolare ci si riferisce all'art. 8, 10 e 12 :\\


\begin{scriptsize}
\textbf{Art. 8 - Cessazione di Licenza}

Non puoi propagare o modificare un programma coperto da questa Licenza in maniera diversa da quanto espressamente consentito da questa Licenza. Qualunque tentativo di propagare o modificare altrimenti il Programma è nullo, e provoca l'immediata cessazione dei diritti garantiti da questa Licenza (\textbf{compresi tutte le eventuali licenze di brevetto garantite ai sensi del terzo paragrafo della sezione 11}).
[...]\\

\textbf{Art. 10 - Licenza Automatica per i successivi destinatari}

[...]
Non puoi imporre nessuna ulteriore restrizione sull'esercizio dei diritti garantiti o affermati da questa Licenza. \textbf{Per esempio, non puoi imporre un prezzo di licenza, una royalty}, o altri costi per l'esercizio dei diritti garantiti da questa Licenza, e non puoi dar corso ad una controversia (ivi incluse le controversie incrociate o la difesa in cause legali) affermando che \textbf{siano stati violati dei brevetti} a causa della produzione, dell'uso, della vendita, della messa in vendita o dell'importazione del Programma o di sue parti.\\

\textbf{Art. 12 - Nessuna resa di libertà altrui}


Se ti vengono imposte delle condizioni (da un ordine giudiziario, da un accordo o da qualunque altra eventualità) che contraddicono le condizioni di questa Licenza, non sei in nessun modo esonerato dal rispetto delle condizioni di questa Licenza. Se non puoi distribuire un Programma coperto da questa Licenza per sottostare simultaneamente agli obblighi derivanti da questa Licenza e ad altri obblighi pertinenti, allora non puoi distribuire il Programma per nessun motivo. \textbf{Per esempio, se accetti delle condizioni che ti obbligano a richiedere il pagamento di una royalty per le distribuzioni successivamente effettuate da coloro ai quali hai distribuito il Programma, l'unico modo per soddisfare sia queste condizioni che questa Licenza è evitare del tutto la distribuzione del Programma.}\\


\end{scriptsize}


Come si nota dall'articolo 8 sulla cessazione della licenza in questa versione si prevede non solo la cessazione dei classici diritti d'autore ma anche le eventuali licenze di brevetto che la GPLv2 non ratificava. E ancora l'articolo 10 ribadisce che non si possono applicare royalty al software suddetto nè tanto meno è permesso dar luogo a processi o cause per la violazione dei brevetti a riguardo del software sotto licenza. Si ribadisce concludendo nell'articolo 12 che l'unico modo per sottoscrivere la licenza e comunque autorizzare il pagamento di royalty corrisponde chiaramente alla non distribuzione del programma. Anche in questo caso, sapientemente, non si vieta \textit{in toto} le royalty, ma indirettamente se ne vieta l'uso.\\

La parte di negazione dei brevetti si ratifica con l'art.11 che afferma invece:\\

\begin{scriptsize}

\textbf{Art. 11 -  Brevetti}
[...]
Se distribuisci un programma coperto da questa Licenza, confidando consapevolmente su una licenza di brevetto, e il Sorgente Corrispondente per il programma non è reso disponibile per la copia, senza alcun onere aggiuntivo e comunque nel rispetto delle condizioni di questa Licenza, attraverso un server di rete pubblicamente accessibile o tramite altri mezzi facilmente accessibili, allora devi (1) \textbf{fare in modo che il Sorgente Corrispondente sia reso disponibile come sopra}, oppure (2) \textbf{fare in modo di rinunciare ai benefici della licenza di brevetto per quel particolare programma}, oppure (3) adoperarti, in maniera consistente con le condizioni di questa Licenza, \textbf{per estendere la licenza di brevetto a tutti i destinatari successivi}. "Confidare consapevolmente" significa che tu sei attualmente cosciente che, eccettuata la licenza di brevetto, la distribuzione da parte tua di un programma protetto da questa Licenza in un paese, o l'utilizzo in un paese del programma coperto da questa Licenza da parte di un destinatario, può violare uno o più brevetti in quel paese che tu hai ragione di ritenere validi.\\

\end{scriptsize}


\`E infatti con questo nuovo art. 11 (Licenza sui brevetti) che si impone al titolare di un brevetto che lo utilizza in un software regolato dalla GPL3, di concedere agli utenti licenza gratuita e non esclusiva sul brevetto in questione, rinunciando di fatto al monopolio. La scelta risulta piuttosto drastica e tagliente: basta pensare al caso di un ingegnere che fosse titolare di un brevetto e che potrebbe trovarsi nella condizione di non scegliere la GPL3, perché questa priverebbe di valore economico la sua creazione. \`E però anche vero che questo esempio coincide proprio il risultato che questa licenza vuole raggiungere contro i software patents.


In sintesi la GPLv3, aggiunge alla vecchia licenza, oltre che la valenza di copyleft virale anche:

\begin{itemize}
	\item  la difesa da tecniche di restrizione dell'accesso al dispositivo
	\item  la difesa da forma di accordo per tutelare da infrazioni di brevetto.
	\item  l'eliminazione di qualsiasi forma di brevetto dal software libero o in caso vi sia uno di proprietà dell'autore, la cessazione in automatica dei diritti a tutti i fruitori del programma.
\end{itemize}

Per questo motivi la licenza GPLv3 è stata etichettata da molti esponenti del movimento opensource, di minor rilvenza etica rispetto alle idee di FSF, una licenza non propositiva ma in difesa di alcune tecnlogie potrebbero dal loro punto di vista monopolizzare il mondo della produzione software a beneficio delle imprese detentrici di brevetto.





\section{Un esempio di brevetto vincolato dalla GPLv3}

In questa sezione si cerca di dare una valenza pratica a quanto analizzato e osservato nella precedente. In particolare sarà fornito un esempio di software libero aderente alla nuova licenza e saranno riportate le conseguenze di tale adozione in materia di software patens.

Dopo il rilascio della GPLv3,solo 122 progetti hanno deciso di adottare la nuova licenza e solo 3 si sono convertiti alla LGPLv3. \footnote{Sono i dati riportati dalla società Palamida che sta tentando di monitorare la reazione del mondo FLOSS dall’introduzione della nuova licenza al Luglio 2007}


Dal 9 Luglio 2007 però a questa schiera di progetti si è aggiunto uno nuovo e determinante, quello di Samba. Per meglio capire il progetto Samba, si affronta brevemente la storia del protocollo di condivisione su rete SMB, che Samba implementa.

SMB è stato inventato da Barry Feigenbaum presso la IBM, ma la versione più largamente usata è stata pesantemente modificata da Microsoft. A causa dell'importanza del protocollo SMB, necessario all'interoperabilita' con la piattaforma Microsoft Windows, nacque il progetto Samba, che rappresenta una implementazione free usata per garantire compatibilita' SMB con sistemi operativi non Microsoft e Unix in generale come GNU/Linux o Apple.

Da ciò si può capire l'importanza del progetto nell'interoperabilita tra sistemi operativi ed in questo senso, la spiegazione della storia del protocollo, è stata fondamentale per capire l'importanza di un cambiamento di licenza di questo programma.

Infatti il team di sviluppo della suite in grado di far dialogare il mondo Windows con i sistemi Unix ha preso la decisione di passare alla GPLv3 nelle prossime versioni.
Per evidenziare l’avvenimento gli sviluppatori hanno deciso di cambiare numero di versione passando dalla 3.0.25 alla 3.2.0.
Quest’ultima sarà la prima release ad essere rilasciata sotto GPLv3 mentre tutte le precedenti versioni rimarranno sotto GPLv2.

Leggendo le FAQ ufficiale del sito reperibili all'indirizzo \\ \textit{http://news.samba.org/announcements/samba\_gplv3/} si notano i cambiamenti avventui con l'avvento della nuova licenza. I cambiamente possono essere anche scontati, ma la loro rilevanza è gigantesca, considerando anche il recente accordo per la tutela dei brevetti che società come Novell ha recentemente stabilito con Microsoft.


\begin{figure}[h]
	\begin{center}
		\includegraphics[scale=0.45]{figure/samba.png}
	\end{center}
	\caption{\textit{Il logo della suite software Samba}}
\end{figure}

Da quanto emerge dalle FAQ vi è l’esplicito divieto di includere la nuova versione della suite 3.2.0 per le distribuzioni e i vendor, come ad esempio Novell, che hanno stipulato accordi sui brevetti chiamati anche \textit{Patent covenant deals} contrari a quanto indicato nella GPLv3 nell'articolo 11.\\
Dal lato propositivo come il team di Samba ha assicurato che la nuova GPLv3 è adotta poiché punta a migliorare la compatibilità con altre licenze, rendendo più semplice l'adozione a livello internazionale.

\newpage

Per convalidare la dimostrazione della validità della GPLv3 riportiamo l'ultima ma più decisvo punto riportato dal sito web di Samba riguardante i brevetti software:\\





\begin{footnotesize}
%``What about patent covenant agreements ? How do they affect the distribution of Samba?''

\textbf{``Cosa cambia con la nuova licenza rispetto agli accordi sull'infrazione dei brevetti? Questo come si ripercuoterà sulla distribuzione di Samba?''}\\

``I brevetti sono esplicitamente incompatibili con la licenza, se sono brevetti discriminatori come sottolineato nell'art. 11 di essa. I distributori di Samba che hanno sottoscritto tali accordi non hanno il diritto di distribuire alcuna versione di Samba coperta dalla GPLv3 (Samba 3.2 o sucessivi). Comunque i diritti dei vendors di rilasciare le versione precedenti sono rimasti gli stessi, cioè tutelati dalla licenza GPLv2. Se ci sono dubbi, si è consigliati di consultare uno studio legale.''\\

%``Patent covenant deals done after 28 March 2007 are explicitly incompatible with the license if they are "discriminatory" under section 11 of the GPLv3. Samba distributors who have made such patent covenant agreements after that date will not have the right to distribute any version of Samba covered by the GPLv3 (Samba 3.2 or later). The rights of vendors to ship 3.0.25b and previous versions is unchanged and remains as it was under the GPLv2. Consult legal advice if you are in doubt.''
\end{footnotesize}

