 \chapter{Il software libero nel sistema giuridico informatico}

Appurate le definizioni di forma di tutela della Proprietà Intellettuale come copyright, brevetto e marchio nel capitolo 1 e la storia dei brevetti applicati software nei mercati più sviluppati come USA e UE nel capitolo 2, si sfrutta i seguenti due capitolo dell'elaborato per approfondire il dibattito sulla tutela del software che oscilla tra copyrigt e brevetti.

In particolare nel capitlo 3 si approfondirà come il movimento opensource stia cercando di difendersi dalla brevettazione del software attraverso la licenza madre copyleft \textit{par excellence}: la GPL, modificata di recente per questo motivo approdando alla terza versione.
In questo trattato non si approfondiranno nè i concetti di opensource o free software nè il concetto di copyleft, che invece possono essere conosciuti rispettivamente attraverso la lettura di \cite[Compendio di libertà informatica e cultura open]{Aliprandi-compendio} e \cite[Copyleft e Opencontent]{Aliprandi-copyleft}

Il capitolo 4 invece sarà più pratico: si metteranno in luce delle conseguenze che la brevettazione software apporta sia nel software libero che in quello proprietario e commerciale; quindi si farà un esempio di brevetto, verrà spiegato e si vedrà le conseguenze portati in USA e in UE.

Adesso si va ad approfondire alla luce di quanto visto fino ad ora, come si tutela il \textit{free software} nel sistema giuridico informatico.


\section{Le licenze libere}
Breve introduzione
	\subsection{La licenza GPL}
	breve
\section{L'approccio ai brevetti della nuova licenza GPLv3}
va letta la GPL3 e spulciata per quanto riguarda i brevetti riportando magari qualche tratto saliente.
\section{Un esempio di brevetto vincolato dalla GPLv3}
idem

