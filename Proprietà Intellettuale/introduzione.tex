 \chapter{Introduzione}
L'invenzione è l'anima motrice dello sviluppo. Ogni singolo oggetto che al momento stiamo utilizzando è frutto del lavoro inventivo di qualcuno, che ha messo a disposizione dell'umanità la sua genialità, consentendo lo sviluppo del progresso. 

La possibilità che ogni individuo ha di concepire liberamente invenzioni è una facoltà che necessariamente deve essere tutelata, in modo da garantire che gli sforzi fatti per raggiungere l'obiettivo non siano poi utilizzati abusivamente da terzi, totalmente al di fuori del processo di sviluppo. 

Vedremo in questa tesina gli strumenti che il campo industriale offre per la tutela giuridica ed economica alle innovazioni, analizzando inizialmente tutte le possibilità offerte, in parallelo a quanto affrontato nel corso di Gestione e Conoscenza della Proprietà Intellettuale tenuto dal prof. Cascini nella Facoltà di Ingegneria di Firenze. 

Successivamente sposteremo il focus su un campo dove la tutela dell'innovazione è al centro di un attualissimo dibattito ed è fronte di imponenti scontri economici, rafforzati anche dall'approccio opposto tenuto dall'Unione Europea rispetto a quello tenuto dagli Stati Uniti: il campo informatico;  saranno a questo punto trattate tutte le alternative offerte nelle due legislazioni di riferimento (USA e UE), contemplando le differenze che possono intercorrere tra l'una e l'altra e spiegando l'attuale legislazione riguardante i brevetti nell'una e nell'altra casistica.

Per rendere attuale la questione, dopo la trattazione teorico/giuridica saranno affrontati due casi pratici di altissimo interesse attuale. Sarà innanzitutto introdotto il concetto di software libero, con una introduzione al sistema di licenze software ideato dalla Free Software Foundation, per arrivare a trattare le spinose questioni che riguardano le clausole della nuova versione licenza GPLv3\cite{gpl} riguardanti appunto il campo brevettuale. Andremo ad analizzare come in base all'attuazione di questa licenza andranno a modificarsi gli equilibri di mercato in campo software, cercando di analizzare i pro e i contro che può indurre al campo informatico.

L'ultima questione porterà il problema sul piano pratico, affrontando una casistica tecnica riferita all'anomalia legislativa USA/UE sui brevetti, cercando di rendere chiaro come la differenza di legislazione possa indurre gli sviluppatori a scelte a volte incompatibili con il normale ciclo di sviluppo di un software, e come possa pesare l'impatto economico di alcune aziende sul mercato.
