%%%%%%%%%%%%%%%%%%%%%%%%%%%%%%%%%%%%%%%%%%
\section{Model Based Tracking}

\frame{\frametitle{Tracking Model Based}


I due filtri possono tracciare oggetti di qualsiasi natur, ma necessitano di un modello che definisca il moto dell'oggetto studiato. \\

\begin{block}{Definizione}
Per modello si intende la rappresentazione di un oggetto che trovi corrispondenza col fenomeno modellato per il fatto di riprodurne le caratteristiche e i comportamenti fondamentali.
\end{block}

Il modello può essere ricavato empiricamente oppure conosciuto a priori.\\

Nel caso più generale possibile è possibile utilizzare la legge del moto di Newton:

\begin{equation}
s(t) = s_0 \, + \, u \, t + \frac{1}{2} a \, t^2
\end{equation}
}
%%%%%%%%%%%%%%%%%%%%%%%%%%%%%%%%%%%%%%%%%%

\subsection{Kalman Filter}

\frame{\frametitle{Kalman Filter}

\begin{block}{}
Il filtro di Kalman è un insieme di equazioni matematiche che si offre come strumento per la stima dello stato di un sistema dinamico, sulla base di misure soggette a rumore, anche quando la vera natura del sistema è sconosciuta. 
\end{block}
E' lo strumento più utilizzato nei problemi di tracciamento, anche se dimostra veramente efficiente solo nei casi in cui:

\begin{itemize}
\item il moto è molto semplice (lineare)
\item l'oggetto da tracciare possa essere rappresentato come un punto in movimento 
\item il rumore che incide sul sistema possa essere ricondotto a rumore di tipo gaussiano.
\end{itemize}

}

\frame{\frametitle{Le equazioni}

\begin{block}{}
\begin{center}
\begin{cases}
\hat x_t = A \hat x_{t-1} + B u_{t-1} + w_t\\

z_t = H x_t + v_t\\

\hat x_{t+1} = \hat x_t + K_t \, (z_t - H \hat x_t)
\end{cases}
\end{center}
\end{block}

\begin{footnotesize}
\begin{itemize}
\item $\hat x_t$ è lo stato dell'oggetto al tempo t secondo il modello da noi definito
\item $x_t$ è lo stato reale direttamente misurato dell'oggetto
\item $z_t$ è la scelta dei parametri misurati che riteniamo utile a descrivere il moto tenuto conto anche un certo errore sulla misura
\item $w_t$ e $v_t$ sono due processi gaussiani con media zero e covariaza rispettivamente Q e R
\item $\hat x_{t+1}$ è la posizione predetta dell'oggetto
\end{itemize}
\end{footnotesize}

}

\frame{\frametitle{Algoritmo}

Le equazioni del filtro di Kalman sono raggruppabili in due macrocategorie associate a due momenti ben distinti dell'algoritmo di predizione:
\begin{figure}[hb]
\centering
	\includegraphics[scale=0.35]{../relazione/figure/cicloKalman.png}
\caption{\tiny \textit{Ciclo di Kalman completo}\label{fig:completeKalman}}
\end{figure}

}

\frame{\frametitle{Il Modello - 1}
Per descrivere il moto dei nostri oggetti ci è sembrato congeniale rappresentare il
generico moto di un punto nel piano.

\begin{figure}[hb]
\centering
	\includegraphics[scale=0.3]{../relazione/figure/motopiano.png}
\caption{\textit{Esempio di vettori di stato moto sul piano piano}\label{fig:motopiano}}
\end{figure}

\begin{footnotesize}
\begin{itemize}
\item $(x,y)$ è la posizione data secondo le coordinate 
\item $(v_x,v_y)$ è la velocità ripespettivamente orizzontale e verticale dell'oggetto nel punto $(x,y)$ 
\end{itemize}
\end{footnotesize}
}


\frame{\frametitle{Il Modello - 2}
\begin{itemize}
\item \begin{math} \hat x= \begin{bmatrix} x \\ y \\ v_{x} \\ v_{y} \end{bmatrix} \end{math}  \footnotesize{è il vettore di stato}\\
\end{itemize}

\begin{itemize}
\item $A = \begin{bmatrix} 1 & 0 & \Delta_{t} & 0 \\ 0 & 1 & 0 & \Delta_{t} \\ 0 & 0 & 1 & 0 \\ 0 & 0 & 0 & 1 \end{bmatrix}$ \footnotesize{è la matrice di transizione del modello}\\
\end{itemize}

\begin{itemize}
\item $B u_{t} = 0$ \footnotesize{sull'oggetto non agiscono forze esterne}
\end{itemize}
}

\frame{\frametitle{Il modello - 3}
\begin{columns}
\column{.50 \textwidth}
\begin{itemize}
\item $Q = \begin{bmatrix} \rho & 0 & 0 & 0 \\ 0 & \rho & 0 & 0 \\ 0 & 0 & \rho & 0 \\ 0 & 0 & 0 & \rho  \end{bmatrix}$ 
\end{itemize}
\column{.50 \textwidth}
\begin{footnotesize}covarianza del processo che rappresenta il rumore sul sistema\end{footnotesize}
\end{columns}

\begin{itemize}
\item $H = \begin{bmatrix} 1 & 0 & 0 & 0 \\ 0 & 1 & 0 & 0 \end{bmatrix}$
\end{itemize}

\begin{itemize}
\item $R = \begin{bmatrix} 0.285 & 0.005 \\ 0.005 & 0.046 \end{bmatrix}$ \begin{footnotesize}definisce il rumore associato alla misura\end{footnotesize}
\end{itemize}
}

\subsection{Condesation}

\frame{\frametitle{ConDenSation Satisfaction}

\begin{block}{Conditional Density Propagation}
\begin{itemize}
\item E' un algoritmo di tipo probabilistico che risulta molto robusto rispetto a dati rumorosi e a cambiamenti di stato non lineari.
 
\item Permette di essere utilizzato per lo studio di moti descritti anche da modelli complessi di tipo non lineare.

\item L'algoritmo utilizza un campionamento casuale e ordinato delle posizioni assunte dall'oggetto nei vari istanti di tempo per modellare funzioni di densità di probabilità arbitrariamente complesse. 
\end{itemize}
\end{block}

}

\frame{\frametitle{Condensation Illusion}

Utilizza un numero N finito di campioni per approssimare la curva che descrive la distribuzione dei dati.\\~\\
Ciascun campione -  sample - consiste di due valori: lo stato e il peso.

\begin{block}{}
Chiamiamo con $H_{t}$ il vettore dei samples all'istante $t$:
\begin{center}
$H_t = \{\overrightarrow{s_1}(t), ... ,\overrightarrow{s_N}(t)\}$
\end{center}

Dove:
\begin{itemize}
\item $\overrightarrow{s_i}(t)= \{\overrightarrow{x_i}(t),p(x_i(t))\}$
\item $\overrightarrow{x_i}(t)$ è la posizione stimata per il sample $i$ all'istante $t$.
\item $p(\overrightarrow{x_i}(t))$ è la probabilità associata alla posizione $\overrightarrow{x_i}(t)$ che caratterizza il sample $i$
\end{itemize}
\end{block}

}

\frame{\frametitle{Algoritmo - Inizializzazione}
Al primo passo dell'algoritmo si inizializza tutti i samples:
\begin{itemize}
\item Ciascuna posizione può essere scelta in modo casuale secondo una distibuzione uniforme.
\item La probabilità associata a ciascun sample è invece distribuita secondo una gaussiana standard centrata nel valore medio tra il valore massimo e il valore minimo assumibile per la posizione dell'oggetto e la relativa varianza.
\end{itemize}

}


\frame{\frametitle{Algoritmo - passo t}

\begin{itemize}
\item Il sample $\overrightarrow{s_i}(t)$ con probabilità maggiore è la predizione per il Condensation al passo t.\\
\end{itemize}

Per passare dal vettore $H_{t}$ al vettore $H_{t+1}$ si eseguono questi passi:
\begin{block}{}
\begin{small}
\begin{enumerate}
\item Si campiona la posizione reale dell'oggetto: $\overrightarrow{z}(t)$
\item Si calcola la posizione per ciascun sample secondo lo spostamento dato dal modello dinamico che in qualche modo descrive il moto dell'oggetto: 
\begin{equation}
	\overrightarrow{x_i}(t+1)= f(\overrightarrow{x_i}(t))
\end{equation}
\item Si stima la probabilità $p(\overrightarrow{z}(t))$ secondo la densità di probabilità che caratterizza i campioni all'istante t.
\item Per ogni sample è ricalcolata la probabilità condizionata applicando il teorema di Bayes:
\begin{equation}
	p_i(\overrightarrow{x_i}(t+1))= p(\overrightarrow{x_i}(t)\ \mid \overrightarrow{z}(t))
\end{equation}
\end{enumerate}
\end{small}
\end{block}

}


\frame{\frametitle{Il Modello - 4}
La nostra implementazione del Condensation rispetta fedelmete l'algoritmo che è stato sopra presentato.\\~\\
Per quanto riguarda il modello dinamico associato si è utilizzata la stessa equazione valida per il tracciamento fatto con il filtro di Kalman:
\begin{block}{}
\begin{small}
\begin{equation}
\hat x_{t+1} = A \hat x_t
\end{equation}
Dove:
\begin{center}
$A = \begin{bmatrix} 1 & 0 & \Delta_{t} & 0 \\ 0 & 1 & 0 & \Delta_{t} \\ 0 & 0 & 1 & 0 \\ 0 & 0 & 0 & 1 \end{bmatrix} $
\end{center}
\end{small}
\end{block}

}