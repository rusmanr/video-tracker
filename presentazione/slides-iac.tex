%%%%%%%%%%%%%%%%%%%%%%%%%%%%%%%%%%%%%%%%%%
\section{Introduzione}

\frame{\frametitle{Outline}
\tableofcontents
}
%%%%%%%%%%%%%%%%%%%%%%%%%%%%%%%%%%%%%%%%%%

%% MOSTRARE VIDEO AL PROF

%%%%%%%%%%%%%%%%%%%%%%%%%%%%%%%%%%%%%%%%%%


\frame{\frametitle{Introduzione}



\begin{block}{Idea di base del tracking video}
\begin{itemize}
 \item Seguire oggetti- \textit{blob} -  che si muovono per un periodo di tempo, potenzialmente anche lungo.
 \item \'E possibile conoscere:
			\begin{enumerate}
			\item dove un oggetto è stato
			\item è attualmente
			\item anche \textbf{prevedere}  dove sarà
			\end{enumerate}


\end{itemize}
\end{block}



\begin{block}{Ambiti di Utilizzo}
\begin{itemize}
\item Industria per la localizzazione di oggetti in movimento (es. Radar )
\item Sistemi di video sorveglianza intelligente.
\end{itemize}
\end{block}


}
%%%%%%%%%%%%%%%%%%%%%%%%%%%%%%%%%%%%%%%%%%

%%%%%%%%%%%%%%%%%%%%%%%%%%%%%%%%%%%%%%%%%%


\frame{\frametitle{Obiettivi}




\begin{enumerate}
\item Eseguire Tracking basato su modelli tramite:
\begin{itemize}
 \item Kalman Filter
\item ConDensation
\end{itemize}

 \item  Scelta del blob da tracciare in caso di \textbf{tracking multipo}.
 \item  Tracciare a video l'andamento dei due algoritmi, evidenziandoli con colori differenti.;
% \item  Visualizzare un' ellissi per ogni algoritmo che indichi la varianza del vettore di stato per quel tipo di tracking.
 \item Fornire un output dei risulati al fine di ottenere una \textbf{rappresentazione grafica} dell'accuratezza dei due metodi.
 \item Progettare e realizzare l'applicazione in maniera tale che possa essere compilata ed eseguite su \textbf{piattaforme diverse (Win32, Linux)}.
\end{enumerate}



}
%%%%%%%%%%%%%%%%%%%%%%%%%%%%%%%%%%%%%%%%%%

%%%%%%%%%%%%%%%%%%%%%%%%%%%%%%%%%%%%%%%%%%


\frame{\frametitle{Ambiente di lavoro}



\begin{block}{Condizioni Ottimali di Lavoro}
\begin{itemize}
 \item La misura del centroide blob è ottenibile frame per frame ...
 \item ...cosa che non accade nei sistemi non ideali \begin{small}(per simulare cià si è utilizzato il parametro \textbf{MOD})                                                                                                                    \end{small} %fare esempio Tracciamento punto con Kalmans
\end{itemize}
\end{block}

\begin{center}
 \begin{Large}Video \end{Large}
\end{center}
\begin{center}
% use packages: array
\begin{tabular}{|l|l|l|l|} \hline
 & Movie12 & TappetoNoMod & SingleCar \\  \hline
Formato & mjpeg/xvid & avi/xvid & avi/xvid \\ \hline
fps & 25 & 10 & 30 \\ \hline
Durata & 50.4 s & 60 s & 33 s \\ \hline
\end{tabular}
\end{center}
%\begin{block}{Requisiti}
\begin{center}
 \begin{large}Requisiti dei Video \end{large}
\end{center}

\begin{enumerate}
	\item Numero determinato di frame con il background iniziale fisso
	%\item Lo sfondo non vari drasticamente nella ripresa 
	\item Telecamera di ripresa fissa
\end{enumerate}
%\end{block}



%\begin{description}
% \item[Object Tracking da camera fissa] \'E il nostro caso in cui si fa \textit{car tracking} da telecamera fissa
% \item[Object Tracking  da camera che zooma e ruota] \'E il caso del \textit{football player tracking}.
% \item[Active Tracking] \'E il caso del \textit{active face tracking}.
% \end{description}


}
%%%%%%%%%%%%%%%%%%%%%%%%%%%%%%%%%%%%%%%%%%



%%%%%%%%%%%%%%%%%%%%%%%%%%%%%%%%%%%%%%%%%%


\frame{\frametitle{Ground Truth}


%#  Ground Truth
\begin{block}{Blob Detection}
\begin{enumerate}
\item Segmentazione del background tramite Background Subtraction di tipo MoG
\item Maschera binaria foreground/background
\item I blobs sono identificati e filtrati sulla base della loro Area e Vicinanza. %inserire i valori giusti -approfondire
\item Il blob selezionato viene scelto sulla base della distanza euclidea calcolata dal click dell'utente.
\end{enumerate}
 
\end{block}

   % * Ottenimento della misura/Segmentazione fg-bg
   %* Individuazione Blob di interesse (classificando la dimensione, distanza, confronto tra frames)
   % * In sintesti si parla di misura del moto e campionamento dei valori di esso


}
%%%%%%%%%%%%%%%%%%%%%%%%%%%%%%%%%%%%%%%%%%


%%%%%%%%%%%%%%%%%%%%%%%%%%%%%%%%%%%%%%%%%%


\frame{\frametitle{Model-based Tracking}


Uso dei due metodi più importanti:
\begin{itemize}
\item Filtro di Kalman (Anni '50)
\item ConDensation (Anni '90)
\end{itemize}


    %*   Dati Campionati
    %* Introduzione agli algoritmi di predizione fatti basati su modelli
    %* Cenni alle altre tipologie
    %* Come si predice il moto sulla base dei valori campionati


}
%%%%%%%%%%%%%%%%%%%%%%%%%%%%%%%%%%%%%%%%%%


%%%%%%%%%%%%%%%%%%%%%%%%%%%%%%%%%%%%%%%%%%


\frame{\frametitle{Background Subtraction}


\begin{block}{Background subtraction MoG}
\underline{Mixture of Gaussian}
\begin{itemize}
\item efficiente anche su video complessi
\item \textit{tuning} dei parametri come:
\begin{itemize}
 \item Soglia di classificazione
\item Numero di Gaussiane per pixel
\end{itemize}



\end{itemize}
\end{block}


          %*   MoG (Mixture of Gaussian)
          %* Cenno agli altri tipi di segmentazione background

\begin{block}{Altre tipologie}
\begin{itemize}
 \item Distribuzione Unimodale
 \item Tecniche Non Parametriche
 \item Approccio basato su regioni o frame
\end{itemize}

\end{block}



}
%%%%%%%%%%%%%%%%%%%%%%%%%%%%%%%%%%%%%%%%%%
